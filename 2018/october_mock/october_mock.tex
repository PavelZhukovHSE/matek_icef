\documentclass[12pt]{article} % размер шрифта

\usepackage{tikz} % картинки в tikz
\usepackage{microtype} % свешивание пунктуации

\usepackage{array} % для столбцов фиксированной ширины

\usepackage{url} % для вставки ссылок \url{...}

\usepackage{indentfirst} % отступ в первом параграфе

\usepackage{sectsty} % для центрирования названий частей
\allsectionsfont{\centering} % приказываем центрировать все sections

\usepackage{amsthm} % теоремы и доказательства

\theoremstyle{definition} % прямой шрифт в условии теорем
\newtheorem{theorem}{Теорема}[section]


\usepackage{amsmath, amssymb} % куча стандартных математических плюшек

\usepackage[top=2cm, left=1.5cm, right=1.5cm, bottom=2cm]{geometry} % размер текста на странице

\usepackage{lastpage} % чтобы узнать номер последней страницы

\usepackage{enumitem} % дополнительные плюшки для списков
%  например \begin{enumerate}[resume] позволяет продолжить нумерацию в новом списке
\usepackage{caption} % подписи к картинкам без плавающего окружения figure


\usepackage{fancyhdr} % весёлые колонтитулы
\pagestyle{fancy}
\lhead{Math for economists}
\chead{Section A}
\rhead{2018-10-26, ICEF}
\lfoot{Variant $\mu$}
\cfoot{Good luck!}
\rfoot{Total time: 120 min}
\renewcommand{\headrulewidth}{0.4pt}
\renewcommand{\footrulewidth}{0.4pt}



\usepackage{todonotes} % для вставки в документ заметок о том, что осталось сделать
% \todo{Здесь надо коэффициенты исправить}
% \missingfigure{Здесь будет картина Последний день Помпеи}
% команда \listoftodos — печатает все поставленные \todo'шки

\usepackage{booktabs} % красивые таблицы
% заповеди из документации:
% 1. Не используйте вертикальные линии
% 2. Не используйте двойные линии
% 3. Единицы измерения помещайте в шапку таблицы
% 4. Не сокращайте .1 вместо 0.1
% 5. Повторяющееся значение повторяйте, а не говорите "то же"

\usepackage{fontspec} % поддержка разных шрифтов
\usepackage{polyglossia} % поддержка разных языков

\setmainlanguage{english}
\setotherlanguages{russian}

\setmainfont{Linux Libertine O} % выбираем шрифт
% если Linux Libertine не установлен, то
% можно также попробовать Helvetica, Arial, Cambria и т.Д.

% чтобы использовать шрифт Linux Libertine на личном компе,
% его надо предварительно скачать по ссылке
% http://www.linuxlibertine.org/index.php?id=91&L=1

% на сервисах типа sharelatex.com этот шрифт есть :)

\newfontfamily{\cyrillicfonttt}{Linux Libertine O}
% пояснение зачем нужно шаманство с \newfontfamily
% http://tex.stackexchange.com/questions/91507/

\AddEnumerateCounter{\asbuk}{\russian@alph}{щ} % для списков с русскими буквами
%\setlist[enumerate, 2]{label=\asbuk\cdot),ref=\asbuk\cdot} % списки уровня 2 будут буквами а) б) ...

%% эконометрические и вероятностные сокращения
\DeclareMathOperator{\Cov}{Cov}
\DeclareMathOperator{\Corr}{Corr}
\DeclareMathOperator{\Var}{Var}
\DeclareMathOperator{\E}{E}
\DeclareMathOperator{\grad}{grad}
\def \hb{\hat{\beta}}
\def \hs{\hat{\sigma}}
\def \htheta{\hat{\theta}}
\def \s{\sigma}
\def \hy{\hat{y}}
\def \hY{\hat{Y}}
\def \v1{\vec{1}}
\def \e{\varepsilon}
\def \he{\hat{\e}}
\def \z{z}
\def \hVar{\widehat{\Var}}
\def \hCorr{\widehat{\Corr}}
\def \hCov{\widehat{\Cov}}
\def \cN{\mathcal{N}}

\def \putyourname{\fbox{
    \begin{minipage}{42em}
      Name, group no:\vspace*{3ex}\par
      \noindent\dotfill\vspace{2mm}
    \end{minipage}
  }
}



\begin{document}

\putyourname
\begin{enumerate}
  \item (10 points) Consider the function $f(x, y) = x^3 + y^3 + 2xy$.
  Using the total differential find the approximate value of $f(1.98, 0.99)$.

\newpage
\putyourname
  \item (10 points) Consider the system
  \[
  \begin{cases}
  x^3 + y^3 + z^2 = 3 \\
  x + x^3 + 2y^3x = 4 \\
  \end{cases}
  \]

  \begin{enumerate}
    \item Check whether the functions $z(y)$ and $x(y)$ are defined at a point $(1, 1, 1)$;
    \item Find $z'(y)$ if possible.
 \end{enumerate}

\newpage
\putyourname
 \item (10 points) Consider the function $f(x, y, z) = x^2 + 9y^2 + 2xy + \alpha z^2$.
 \begin{enumerate}
   \item Find the Hesse matrix. Clearly state the Young theorem if you use it.
   \item For each value of $\alpha$ find the definiteness of Hesse matrix.
 \end{enumerate}

\newpage
\putyourname
 \item (10 points) Consider the function $u(x)=f(a, b, c)$, where $a=\alpha(q,r)$, $b=\beta(x)$,
 $c=\gamma(x,q)$, $q=x^2$ and $r=x^3$. All the functions are differentiable.
 Find $u'(x)$.

\newpage
\putyourname
  \item (10 points) Consider the function $f(x, y) = x^2 + y^2 + 4y$.
  The microbe Veniamin is standing at $(1,1)$ and is moving according to a simple rule.
  From a point $(a, b)$ he jumps into the point $(a, b) - 0.01\grad f(a,b)$.
  \begin{enumerate}
    \item Where Veniamin will be after two jumps?
    \item What will be the approximate location of Veniamin after $2018$ jumps?
  \end{enumerate}

\newpage
\putyourname
  \item (10 points) Let $h(a, b) = \int_a^b \exp(-t^2) \cdot dt$. Find the $\grad h(1, 2)$.



\end{enumerate}



\newpage
\chead{Section B}
\putyourname

\begin{enumerate}[resume]
\item The domain of the function $z = xy - \frac{2}{3}x\sqrt{x} - \frac{1}{3}y^3 +5x+3y$
is the nonnegative quadrant $\{x\geq 0, y \geq 0\}$.
\begin{enumerate}
  \item (10 points) Find the equation of the tangent plane to the graph of $z$ at $(1,1,8)$.
  \item  (10 points) Let $\grad z(1,1) = c$. Find all such points that $\grad z(x,y)=c$.
\end{enumerate}

\newpage
\putyourname
\item Two drivers on a lonely island get utility from fast driving and money.
Let $0\leq x_1 \leq 1$ be the speed of the first car
and $0\leq x_2 \leq 1$ be the speed of the second car, respectively.
They have the same amount of wealth $I>1$. Utilities of the drivers are
$U_1(x_1, x_2) = x_1 + I\cdot (1 - x_1 x_2)$ and $U_2(x_1, x_2) = \ln x_2 + I\cdot (1 - x_1 x_2)$.
\begin{enumerate}
  \item (7 points) On $(x_1, x_2)$-plane draw the solutions of
  the equations $\frac{\partial U_1}{\partial x_1}=0$ and $\frac{\partial U_2}{\partial x_2}=0$.
  \item (10 points) Let $(x_1^*, x_2^*)=(1, 1/I)$.
  Show that the system of inequalities hold $U_1(x_1^*, x_2^*)\geq U_1(x_1, x_2^*)$
  and  $U_2(x_1^*, x_2^*)\geq U_2(x_1^*, x_2)$.
  \item (3 points) Explain why even the small bribe
  offered by the second driver will stop the first driver from using his car?
\end{enumerate}

\end{enumerate}






%%%% вар!!!

\newpage
\chead{Section A}
\lfoot{Variant $\rho$}

\putyourname
\begin{enumerate}
  \item (10 points) Consider the function $f(x, y) = x^3 + y^3 + 3xy$.
  Using the total differential find the approximate value of $f(1.98, 0.99)$.

\newpage
\putyourname
  \item (10 points) Consider the system
  \[
  \begin{cases}
  x^3 + y^3 + 2z^2 = 4 \\
  x + x^3 + 2y^3x = 4 \\
  \end{cases}
  \]

  \begin{enumerate}
    \item Check whether the functions $z(y)$ and $x(y)$ are defined at a point $(1, 1, 1)$;
    \item Find $z'(y)$ if possible.
 \end{enumerate}

\newpage
\putyourname
 \item (10 points) Consider the function $f(x, y, z) = x^2 + 10y^2 + 2xy + \alpha z^2$.
 \begin{enumerate}
   \item Find the Hesse matrix. Clearly state the Young theorem if you use it.
   \item For each value of $\alpha$ find the definiteness of Hesse matrix.
 \end{enumerate}

\newpage
\putyourname
 \item (10 points) Consider the function $u(x)=f(a, b, c)$, where $a=\alpha(q,r)$, $b=\beta(x)$,
 $c=\gamma(x,q)$, $q=x^2$ and $r=-x^3$. All the functions are differentiable.
 Find $u'(x)$.

\newpage
\putyourname
  \item (10 points) Consider the function $f(x, y) = x^2 + y^2 + 6y$.
  The microbe Veniamin is standing at $(1,1)$ and is moving according to a simple rule.
  From a point $(a, b)$ he jumps into the point $(a, b) - 0.01\grad f(a,b)$.
  \begin{enumerate}
    \item Where Veniamin will be after two jumps?
    \item What will be the approximate location of Veniamin after $2018$ jumps?
  \end{enumerate}

\newpage
\putyourname
  \item (10 points) Let $h(a, b) = \int_a^b \exp(-2t^2) \cdot dt$. Find the $\grad h(1, 2)$.



\end{enumerate}


\newpage
\chead{Section B}
\putyourname

\begin{enumerate}[resume]
\item The domain of the function $z = xy - \frac{2}{3}x\sqrt{x} - \frac{1}{3}y^3 +5x+3y$
is the nonnegative quadrant $\{x\geq 0, y \geq 0\}$.
\begin{enumerate}
  \item (10 points) Find the equation of the tangent plane to the graph of $z$ at $(1,1,8)$.
  \item  (10 points) Let $\grad z(1,1) = c$. Find all such points that $\grad z(x,y)=c$.
\end{enumerate}

\newpage
\putyourname
\item Two drivers on a lonely island get utility from fast driving and money.
Let $0\leq x_1 \leq 1$ be the speed of the first car
and $0\leq x_2 \leq 1$ be the speed of the second car, respectively.
They have the same amount of wealth $I>1$. Utilities of the drivers are
$U_1(x_1, x_2) = x_1 + I\cdot (1 - x_1 x_2)$ and $U_2(x_1, x_2) = \ln x_2 + I\cdot (1 - x_1 x_2)$.
\begin{enumerate}
  \item (7 points) On $(x_1, x_2)$-plane draw the solutions of
  the equations $\frac{\partial U_1}{\partial x_1}=0$ and $\frac{\partial U_2}{\partial x_2}=0$.
  \item (10 points) Let $(x_1^*, x_2^*)=(1, 1/I)$.
  Show that the system of inequalities hold $U_1(x_1^*, x_2^*)\geq U_1(x_1, x_2^*)$
  and  $U_2(x_1^*, x_2^*)\geq U_2(x_1^*, x_2)$.
  \item (3 points) Explain why even the small bribe
  offered by the second driver will stop the first driver from using his car?
\end{enumerate}

\end{enumerate}





%%%% вар!!!

\newpage
\chead{Section A}
\lfoot{Variant $\xi$}

\putyourname
\begin{enumerate}
  \item (10 points) Consider the function $f(x, y) = x^3 + y^3 + 4xy$.
  Using the total differential find the approximate value of $f(1.98, 0.99)$.

\newpage
\putyourname
  \item (10 points) Consider the system
  \[
  \begin{cases}
  x^3 + y^3 + 3z^2 = 5 \\
  x + x^3 + 2y^3x = 4 \\
  \end{cases}
  \]

  \begin{enumerate}
    \item Check whether the functions $z(y)$ and $x(y)$ are defined at a point $(1, 1, 1)$;
    \item Find $z'(y)$ if possible.
 \end{enumerate}

\newpage
\putyourname
 \item (10 points) Consider the function $f(x, y, z) = x^2 + 11y^2 + 2xy + \alpha z^2$.
 \begin{enumerate}
   \item Find the Hesse matrix. Clearly state the Young theorem if you use it.
   \item For each value of $\alpha$ find the definiteness of Hesse matrix.
 \end{enumerate}

\newpage
\putyourname
 \item (10 points) Consider the function $u(x)=f(a, b, c)$, where $a=\alpha(q,r)$, $b=\beta(x)$,
 $c=\gamma(x,q)$, $q=-x^2$ and $r=x^3$. All the functions are differentiable.
 Find $u'(x)$.

\newpage
\putyourname
  \item (10 points) Consider the function $f(x, y) = x^2 + y^2 + 8y$.
  The microbe Veniamin is standing at $(1,1)$ and is moving according to a simple rule.
  From a point $(a, b)$ he jumps into the point $(a, b) - 0.01\grad f(a,b)$.
  \begin{enumerate}
    \item Where Veniamin will be after two jumps?
    \item What will be the approximate location of Veniamin after $2018$ jumps?
  \end{enumerate}

\newpage
\putyourname
  \item (10 points) Let $h(a, b) = \int_a^b \exp(-3t^2) \cdot dt$. Find the $\grad h(1, 2)$.



\end{enumerate}


\newpage
\chead{Section B}
\putyourname

\begin{enumerate}[resume]
\item The domain of the function $z = xy - \frac{2}{3}x\sqrt{x} - \frac{1}{3}y^3 +5x+3y$
is the nonnegative quadrant $\{x\geq 0, y \geq 0\}$.
\begin{enumerate}
  \item (10 points) Find the equation of the tangent plane to the graph of $z$ at $(1,1,8)$.
  \item  (10 points) Let $\grad z(1,1) = c$. Find all such points that $\grad z(x,y)=c$.
\end{enumerate}

\newpage
\putyourname
\item Two drivers on a lonely island get utility from fast driving and money.
Let $0\leq x_1 \leq 1$ be the speed of the first car
and $0\leq x_2 \leq 1$ be the speed of the second car, respectively.
They have the same amount of wealth $I>1$. Utilities of the drivers are
$U_1(x_1, x_2) = x_1 + I\cdot (1 - x_1 x_2)$ and $U_2(x_1, x_2) = \ln x_2 + I\cdot (1 - x_1 x_2)$.
\begin{enumerate}
  \item (7 points) On $(x_1, x_2)$-plane draw the solutions of
  the equations $\frac{\partial U_1}{\partial x_1}=0$ and $\frac{\partial U_2}{\partial x_2}=0$.
  \item (10 points) Let $(x_1^*, x_2^*)=(1, 1/I)$.
  Show that the system of inequalities hold $U_1(x_1^*, x_2^*)\geq U_1(x_1, x_2^*)$
  and  $U_2(x_1^*, x_2^*)\geq U_2(x_1^*, x_2)$.
  \item (3 points) Explain why even the small bribe
  offered by the second driver will stop the first driver from using his car?
\end{enumerate}

\end{enumerate}





%%%% вар!!!

\newpage
\chead{Section A}
\lfoot{Variant $\omega$}

\putyourname
\begin{enumerate}
  \item (10 points) Consider the function $f(x, y) = x^3 + y^3 + 5xy$.
  Using the total differential find the approximate value of $f(1.98, 0.99)$.

\newpage
\putyourname
  \item (10 points) Consider the system
  \[
  \begin{cases}
  x^3 + y^3 + 4z^2 = 6 \\
  x + x^3 + 2y^3x = 4 \\
  \end{cases}
  \]

  \begin{enumerate}
    \item Check whether the functions $z(y)$ and $x(y)$ are defined at a point $(1, 1, 1)$;
    \item Find $z'(y)$ if possible.
 \end{enumerate}

\newpage
\putyourname
 \item (10 points) Consider the function $f(x, y, z) = x^2 + 12y^2 + 2xy + \alpha z^2$.
 \begin{enumerate}
   \item Find the Hesse matrix. Clearly state the Young theorem if you use it.
   \item For each value of $\alpha$ find the definiteness of Hesse matrix.
 \end{enumerate}

\newpage
\putyourname
 \item (10 points) Consider the function $u(x)=f(a, b, c)$, where $a=\alpha(q,r)$, $b=\beta(x)$,
 $c=\gamma(x,q)$, $q=-x^2$ and $r=-x^3$. All the functions are differentiable.
 Find $u'(x)$.

\newpage
\putyourname
  \item (10 points) Consider the function $f(x, y) = x^2 + y^2 + 10y$.
  The microbe Veniamin is standing at $(1,1)$ and is moving according to a simple rule.
  From a point $(a, b)$ he jumps into the point $(a, b) - 0.01\grad f(a,b)$.
  \begin{enumerate}
    \item Where Veniamin will be after two jumps?
    \item What will be the approximate location of Veniamin after $2018$ jumps?
  \end{enumerate}

\newpage
\putyourname
  \item (10 points) Let $h(a, b) = \int_a^b \exp(-4t^2) \cdot dt$. Find the $\grad h(1, 2)$.



\end{enumerate}


\newpage
\chead{Section B}
\putyourname

\begin{enumerate}[resume]
\item The domain of the function $z = xy - \frac{2}{3}x\sqrt{x} - \frac{1}{3}y^3 +5x+3y$
is the nonnegative quadrant $\{x\geq 0, y \geq 0\}$.
\begin{enumerate}
  \item (10 points) Find the equation of the tangent plane to the graph of $z$ at $(1,1,8)$.
  \item  (10 points) Let $\grad z(1,1) = c$. Find all such points that $\grad z(x,y)=c$.
\end{enumerate}

\newpage
\putyourname
\item Two drivers on a lonely island get utility from fast driving and money.
Let $0\leq x_1 \leq 1$ be the speed of the first car
and $0\leq x_2 \leq 1$ be the speed of the second car, respectively.
They have the same amount of wealth $I>1$. Utilities of the drivers are
$U_1(x_1, x_2) = x_1 + I\cdot (1 - x_1 x_2)$ and $U_2(x_1, x_2) = \ln x_2 + I\cdot (1 - x_1 x_2)$.
\begin{enumerate}
  \item (7 points) On $(x_1, x_2)$-plane draw the solutions of
  the equations $\frac{\partial U_1}{\partial x_1}=0$ and $\frac{\partial U_2}{\partial x_2}=0$.
  \item (10 points) Let $(x_1^*, x_2^*)=(1, 1/I)$.
  Show that the system of inequalities hold $U_1(x_1^*, x_2^*)\geq U_1(x_1, x_2^*)$
  and  $U_2(x_1^*, x_2^*)\geq U_2(x_1^*, x_2)$.
  \item (3 points) Explain why even the small bribe
  offered by the second driver will stop the first driver from using his car?
\end{enumerate}

\end{enumerate}
















\end{document}
