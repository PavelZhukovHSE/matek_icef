% arara: xelatex
\documentclass[12pt]{article} % размер шрифта

\usepackage{tikz} % картинки в tikz
\usepackage{microtype} % свешивание пунктуации

\usepackage{array} % для столбцов фиксированной ширины

\usepackage{url} % для вставки ссылок \url{...}

\usepackage{indentfirst} % отступ в первом параграфе

\usepackage{sectsty} % для центрирования названий частей
\allsectionsfont{\centering} % приказываем центрировать все sections

\usepackage{amsthm} % теоремы и доказательства

\theoremstyle{definition} % прямой шрифт в условии теорем
\newtheorem{theorem}{Теорема}[section]


\usepackage{amsmath, amssymb} % куча стандартных математических плюшек

\usepackage[top=2cm, left=1.5cm, right=1.5cm, bottom=2cm]{geometry} % размер текста на странице

\usepackage{lastpage} % чтобы узнать номер последней страницы

\usepackage{enumitem} % дополнительные плюшки для списков
%  например \begin{enumerate}[resume] позволяет продолжить нумерацию в новом списке
\usepackage{caption} % подписи к картинкам без плавающего окружения figure


\usepackage{fancyhdr} % весёлые колонтитулы
\pagestyle{fancy}
\lhead{Math for economists}
\chead{Section A}
\rhead{2021-12-25, ICEF}
\lfoot{Variant $\alpha$}
\cfoot{Good luck!}
\rfoot{}
\renewcommand{\headrulewidth}{0.4pt}
\renewcommand{\footrulewidth}{0.4pt}



\usepackage{todonotes} % для вставки в документ заметок о том, что осталось сделать
% \todo{Здесь надо коэффициенты исправить}
% \missingfigure{Здесь будет картина Последний день Помпеи}
% команда \listoftodos — печатает все поставленные \todo'шки

\usepackage{booktabs} % красивые таблицы
% заповеди из документации:
% 1. Не используйте вертикальные линии
% 2. Не используйте двойные линии
% 3. Единицы измерения помещайте в шапку таблицы
% 4. Не сокращайте .1 вместо 0.1
% 5. Повторяющееся значение повторяйте, а не говорите "то же"

\usepackage{fontspec} % поддержка разных шрифтов
\usepackage{polyglossia} % поддержка разных языков

\setmainlanguage{english}
\setotherlanguages{russian}

\setmainfont{Linux Libertine O} % выбираем шрифт
% если Linux Libertine не установлен, то
% можно также попробовать Helvetica, Arial, Cambria и т.Д.

% чтобы использовать шрифт Linux Libertine на личном компе,
% его надо предварительно скачать по ссылке
% http://www.linuxlibertine.org/index.php?id=91&L=1

% на сервисах типа sharelatex.com этот шрифт есть :)

\newfontfamily{\cyrillicfonttt}{Linux Libertine O}
% пояснение зачем нужно шаманство с \newfontfamily
% http://tex.stackexchange.com/questions/91507/

\AddEnumerateCounter{\asbuk}{\russian@alph}{щ} % для списков с русскими буквами
%\setlist[enumerate, 2]{label=\asbuk\cdot),ref=\asbuk\cdot} % списки уровня 2 будут буквами а) б) ...

%% эконометрические и вероятностные сокращения
\DeclareMathOperator{\Cov}{Cov}
\DeclareMathOperator{\Corr}{Corr}
\DeclareMathOperator{\Var}{Var}
\DeclareMathOperator{\E}{E}
\DeclareMathOperator{\grad}{grad}
\def \hb{\hat{\beta}}
\def \hs{\hat{\sigma}}
\def \htheta{\hat{\theta}}
\def \s{\sigma}
\def \hy{\hat{y}}
\def \hY{\hat{Y}}
\def \v1{\vec{1}}
\def \e{\varepsilon}
\def \he{\hat{\e}}
\def \z{z}
\def \hVar{\widehat{\Var}}
\def \hCorr{\widehat{\Corr}}
\def \hCov{\widehat{\Cov}}
\def \cN{\mathcal{N}}
\def \RR{\mathbb{R}}


\def \putyourname{\fbox{
    \begin{minipage}{42em}
      Name, group no:\vspace*{3ex}\par
      \noindent\dotfill\vspace{2mm}
    \end{minipage}
  }
}



\begin{document}

\putyourname
\begin{enumerate}

\item (10 points) Ded Moroz considers the function $f(x, y) = (1 + 2x + 3y)^{2022}$.

Please, help him to find the second order Taylor approximation of this function at a point $x=0$, $y=0$.

\newpage
\putyourname
\item (10 points) Consider the function 
\[
f(x, y) = \int_0^x 3e^{u^2} \, du  + \int_0^y 2\cos(u^2) \, du.   
\]

Find the gradient $\grad f$ at the point $(0, 0)$.

\newpage
\putyourname
\item (10 points) Consider the system 
  \[
  \begin{cases}
    x + y + z = 2 \\
    2x^2 + 2y^2 = z^2 \\
  \end{cases}.  
  \]
  \begin{enumerate}
    \item Are the functions $x(z)$ and $y(z)$ defined in a neighborhood of the point $A(x=1, y=-1, z=2)$?
    \item Find $dx/dz$ at the point $A$ if possible. 
  \end{enumerate}

\newpage
\putyourname  
\item (10 points)  
The set $S$ is defined by $S = \{(x, y) \in \mathbb{R}^2 \mid 0 \leq y \leq 2- x^2\}$. 
Two rectangles one on the top of the other are inscribed in $S$, 
thus they have the common side and the upper vertices lie on this parabola.
Let $A_1 + A_2$ be the sum of their areas, where $A_1 >0$, $A_2 >0$. 

Consider the maximization problem $A_1 + A_2 \to \max$.
\begin{enumerate}
  \item Solve the maximization problem or show that the maximum does not exist.
  \item Check whether the Weierstrass theorem is applicable.
\end{enumerate}

\newpage
\putyourname
\item (10 points)  
An implicit function is defined by equation $x^2 + y^2 + xy + 2x + 4y = 0$. Find all the point(s)
$(x^*, y^*)$ on the curve represented by this equation, where
\begin{enumerate}
  \item the tangent line to the curve is horizontal with the equation $y = c$;
  % \item this $c$ has the maximum possible value;
  \item conditions of the implicit function theorem are satisfied.
\end{enumerate}

\newpage
\putyourname
\item (10 points)  
Consider the function $g(x) = \sqrt{x-1+\sqrt{x-1+\sqrt{x-1+\sqrt{\cdots}}}}$ defined for $x>1$.

\begin{enumerate}
    \item Express $g^{2}(x)$ through the linear combination of $g(x)$ and $x-1$.
    \item Find $g'(2)$.
\end{enumerate}


%\item (10 points) The turtle starts at the point $(0, 0)$. 
%Every day she continues her way at constant speed in constant direction. 
%Every midnight she drops her speed by 30\% and turns $90^{\circ}$ counterclockwise. 
%On the first day she goes by 1 kilometer in the positive direction of $OX$ axis.

%What is the ultimate destination of the turtle?




\end{enumerate}



\newpage
\chead{Section B}
\putyourname
\begin{enumerate}[resume]

  \item (20 points) The Mean Value Theorem in Calculus claims that given a continuously differentiable function $f(x)$  
  on the closed segment $[x_0; x_0 + \Delta x]$, there exists $0 < \theta < 1$ such that $f(x_0 + \Delta x) - f(x_0) = 
  f'(x_0 + \theta \Delta x) \Delta x$. 

  Prove the version of this theorem for a function of the two variables $z = f(x,y) \in C^1$ in the following form: 
  there exists $0 < \theta < 1$ such that 
  \[
    f(x_0 + \Delta x, y_0 + \Delta y) - f(x_0, y_0) = \frac{\partial f}{\partial x}(x_0 + \theta \Delta x, y_0 + \theta \Delta y)\Delta x + \frac{\partial f}{\partial y}(x_0 + \theta \Delta x, y_0 + \theta \Delta y)\Delta y.  
  \]
  
  
  Hint. Join the 2 points $(x_0, y_0)$ and $(x_0 + \Delta x, y_0 + \Delta y)$ with the segment of a straight line passing through these points 
  and apply the chain rule to the composite function $f(x_0 + t \Delta x, y_0 + t \Delta y)$ where $t\in [0;1]$.
  
  
  \newpage
  \putyourname
    \item An economist solves a problem of the minimization of expenses of an individual whose utility function is $u(x, y) = \sqrt{x} + \sqrt{y}$. 
  Expenses are calculated by the formula $E=3x + 4y$ and the consumer would prefer to fix the value of the utility at 7 utiles.
  \begin{enumerate}
  \item (5 points) Formulate the problem of the expenses minimization and form a Lagrangian of this problem.
  \item (5 points) Using first-order conditions find the minimizing bundle $(x^*, y^*)$.
  \item (5 points) Using bordered Hessian or otherwise check the sufficiency condition.
  \item (5 points) A consumer decided to improve her welfare by adding additional 0,1 to 7 utiles. 
  How this decision will affect the expenditure value $E^* = 3x^* + 4y^*$, 
  where the bundle $(x^*, y^*)$ corresponds to a greater utility value? 
  Use appropriate Envelope Theorem to estimate $E^*$.
\end{enumerate}

\end{enumerate}



\newpage 
\lfoot{Variant $\beta$}
\putyourname
\begin{enumerate}

  \item (10 points) Ded Moroz considers the function $f(x, y) = (1 + 2x + 4y)^{2022}$.
  
  Please, help him to find the second order Taylor approximation of this function at a point $x=0$, $y=0$.
  
  \newpage
\putyourname
\item (10 points) Consider the function 
  \[
  f(x, y) = \int_0^x 3e^{u^2} \, du  + \int_0^y 5\cos(u^2) \, du.   
  \]
  
  Find the gradient $\grad f$ at the point $(0, 0)$.
  

  \newpage
\putyourname
\item (10 points) Consider the system 
    \[
    \begin{cases}
      x + y + z = 2 \\
      2x^2 + 2y^2 = z^2 \\
    \end{cases}.  
    \]
    \begin{enumerate}
      \item Are the functions $x(z)$ and $y(z)$ defined in a neighborhood of the point $A(x=-1, y=1, z=2)$?
      \item Find $dx/dz$ at the point $A$ if possible. 
    \end{enumerate}
  
  
    \newpage
    \putyourname
      \item (10 points)  
  The set $S$ is defined by $S = \{(x, y) \in \mathbb{R}^2 \mid 0 \leq y \leq 3 - x^2\}$. 
  Two rectangles one on the top of the other are inscribed in $S$, 
  thus they have the common side and the upper vertices lie on this parabola.
  Let $A_1 + A_2$ be the sum of their areas, where $A_1 >0$, $A_2 >0$. 
  
  Consider the maximization problem $A_1 + A_2 \to \max$.
  \begin{enumerate}
    \item Solve the maximization problem or show that the maximum does not exist.
    \item Check whether the Weierstrass theorem is applicable.
  \end{enumerate}
  
  \newpage
  \putyourname
    \item (10 points)  
  An implicit function is defined by equation $x^2 + y^2 + xy - 2x - 4y = 0$. Find all the point(s)
  $(x^*, y^*)$ on the curve represented by this equation, where
  \begin{enumerate}
    \item the tangent line to the curve is horizontal with the equation $y = c$;
    % \item this $c$ has the maximum possible value;
    \item conditions of the implicit function theorem are satisfied.
  \end{enumerate}
  
  \newpage
  \putyourname
    \item (10 points)  
  Consider the function $g(x) = \sqrt{x-2+\sqrt{x-2+\sqrt{x-2+\sqrt{\cdots}}}}$ defined for $x>2$.
  
  \begin{enumerate}
      \item Express $g^{2}(x)$ through the linear combination of $g(x)$ and $x-2$.
      \item Find $g'(3)$.
  \end{enumerate}
  
  
  %\item (10 points) The turtle starts at the point $(0, 0)$. 
  %Every day she continues her way at constant speed in constant direction. 
  %Every midnight she drops her speed by 40\% and turns $90^{\circ}$ counterclockwise. 
  %On the first day she goes by 1 kilometer in the positive direction of $OX$ axis.
  
  %What is the ultimate destination of the turtle?
  
  
  
  
  \end{enumerate}
  
  
  
  \newpage
  \chead{Section B}
  \putyourname
  \begin{enumerate}[resume]
  
    \item (20 points) The Mean Value Theorem in Calculus claims that given a continuously differentiable function $f(x)$  
    on the closed segment $[x_0; x_0 + \Delta x]$, there exists $0 < \theta < 1$ such that $f(x_0 + \Delta x) - f(x_0) = 
    f'(x_0 + \theta \Delta x) \Delta x$. 
  
    Prove the version of this theorem for a function of the two variables $z = f(x,y) \in C^1$ in the following form: 
    there exists $0 < \theta < 1$ such that 
    \[
      f(x_0 + \Delta x, y_0 + \Delta y) - f(x_0, y_0) = \frac{\partial f}{\partial x}(x_0 + \theta \Delta x, y_0 + \theta \Delta y)\Delta x + \frac{\partial f}{\partial y}(x_0 + \theta \Delta x, y_0 + \theta \Delta y)\Delta y.  
    \]
    
    
    Hint. Join the 2 points $(x_0, y_0)$ and $(x_0 + \Delta x, y_0 + \Delta y)$ with the segment of a straight line passing through these points 
    and apply the chain rule to the composite function $f(x_0 + t \Delta x, y_0 + t \Delta y)$ where $t\in [0;1]$.
    
    
    \newpage
    \putyourname
      \item An economist solves a problem of the minimization of expenses of an individual whose utility function is $u(x, y) = \sqrt{x} + \sqrt{y}$. 
    Expenses are calculated by the formula $E=4x + 3y$ and the consumer would prefer to fix the value of the utility at 7 utiles.
    \begin{enumerate}
    \item (5 points) Formulate the problem of the expenses minimization and form a Lagrangian of this problem.
    \item (5 points) Using first-order conditions find the minimizing bundle $(x^*, y^*)$.
    \item (5 points) Using bordered Hessian or otherwise check the sufficiency condition.
    \item (5 points) A consumer decided to improve her welfare by adding additional 0,1 to 7 utiles. 
    How this decision will affect the expenditure value $E^* = 4x^* + 3y^*$, 
    where the bundle $(x^*, y^*)$ corresponds to a greater utility value? 
    Use appropriate Envelope Theorem to estimate $E^*$.
  \end{enumerate}
  
  \end{enumerate}
  

  
  \newpage
  \lfoot{Variant $\delta$}
  \putyourname
  \begin{enumerate}

    \item (10 points) Ded Moroz considers the function $f(x, y) = (1 + 4x + 2y)^{2022}$.
    
    Please, help him to find the second order Taylor approximation of this function at a point $x=0$, $y=0$.
    
    \newpage
\putyourname
\item (10 points) Consider the function 
    \[
    f(x, y) = \int_0^x 3e^{u^2} \, du  + \int_0^y 7\cos(u^2) \, du.   
    \]
    
    Find the gradient $\grad f$ at the point $(0, 0)$.
    

    \newpage
    \putyourname
        \item (10 points) Consider the system 
      \[
      \begin{cases}
        x + y + z = 2 \\
        2x^2 + 2y^2 = z^2 \\
      \end{cases}.  
      \]
      \begin{enumerate}
        \item Are the functions $x(z)$ and $y(z)$ defined in a neighborhood of the point $A(x=1, y=-1, z=2)$?
        \item Find $dy/dz$ at the point $A$ if possible. 
      \end{enumerate}
    
    
      \newpage
      \putyourname
          \item (10 points)  
    The set $S$ is defined by $S = \{(x, y) \in \mathbb{R}^2 \mid 0 \leq y \leq 4- x^2\}$. 
    Two rectangles one on the top of the other are inscribed in $S$, 
    thus they have the common side and the upper vertices lie on this parabola.
    Let $A_1 + A_2$ be the sum of their areas, where $A_1 >0$, $A_2 >0$. 
    
    Consider the maximization problem $A_1 + A_2 \to \max$.
    \begin{enumerate}
      \item Solve the maximization problem or show that the maximum does not exist.
      \item Check whether the Weierstrass theorem is applicable.
    \end{enumerate}
    
    \newpage
    \putyourname
        \item (10 points)  
    An implicit function is defined by equation $x^2 + y^2 - xy - 2x + 4y = 0$. Find all the point(s)
    $(x^*, y^*)$ on the curve represented by this equation, where
    \begin{enumerate}
      \item the tangent line to the curve is horizontal with the equation $y = c$;
      % \item this $c$ has the maximum possible value;
      \item conditions of the implicit function theorem are satisfied.
    \end{enumerate}
    
    \newpage
    \putyourname
        \item (10 points)  
    Consider the function $g(x) = \sqrt{x-3+\sqrt{x-3+\sqrt{x-3+\sqrt{\cdots}}}}$ defined for $x>3$.
    
    \begin{enumerate}
        \item Express $g^{2}(x)$ through the linear combination of $g(x)$ and $x-3$.
        \item Find $g'(4)$.
    \end{enumerate}
    
    
    %\item (10 points) The turtle starts at the point $(0, 0)$. 
    %Every day she continues her way at constant speed in constant direction. 
    %Every midnight she drops her speed by 50\% and turns $90^{\circ}$ counterclockwise. 
    %On the first day she goes by 1 kilometer in the positive direction of $OX$ axis.
    
    %What is the ultimate destination of the turtle?
    
    
    
    
    \end{enumerate}
    
    
    
    \newpage
    \chead{Section B}
    \putyourname    
    \begin{enumerate}[resume]
    
      \item (20 points) The Mean Value Theorem in Calculus claims that given a continuously differentiable function $f(x)$  
      on the closed segment $[x_0; x_0 + \Delta x]$, there exists $0 < \theta < 1$ such that $f(x_0 + \Delta x) - f(x_0) = 
      f'(x_0 + \theta \Delta x) \Delta x$. 
    
      Prove the version of this theorem for a function of the two variables $z = f(x,y) \in C^1$ in the following form: 
      there exists $0 < \theta < 1$ such that 
      \[
        f(x_0 + \Delta x, y_0 + \Delta y) - f(x_0, y_0) = \frac{\partial f}{\partial x}(x_0 + \theta \Delta x, y_0 + \theta \Delta y)\Delta x + \frac{\partial f}{\partial y}(x_0 + \theta \Delta x, y_0 + \theta \Delta y)\Delta y.  
      \]
      
      
      Hint. Join the 2 points $(x_0, y_0)$ and $(x_0 + \Delta x, y_0 + \Delta y)$ with the segment of a straight line passing through these points 
      and apply the chain rule to the composite function $f(x_0 + t \Delta x, y_0 + t \Delta y)$ where $t\in [0;1]$.
      
      
      \newpage
      \putyourname
        \item An economist solves a problem of the minimization of expenses of an individual whose utility function is $u(x, y) = \sqrt{x} + \sqrt{y}$. 
      Expenses are calculated by the formula $E=2x + 5y$ and the consumer would prefer to fix the value of the utility at 7 utiles.
      \begin{enumerate}
      \item (5 points) Formulate the problem of the expenses minimization and form a Lagrangian of this problem.
      \item (5 points) Using first-order conditions find the minimizing bundle $(x^*, y^*)$.
      \item (5 points) Using bordered Hessian or otherwise check the sufficiency condition.
      \item (5 points) A consumer decided to improve her welfare by adding additional 0,1 to 7 utiles. 
      How this decision will affect the expenditure value $E^* = 2x^* + 5y^*$, 
      where the bundle $(x^*, y^*)$ corresponds to a greater utility value? 
      Use appropriate Envelope Theorem to estimate $E^*$.
    \end{enumerate}
    
    \end{enumerate}
    

\newpage 
\lfoot{Variant $o$}
\putyourname
\begin{enumerate}

  \item (10 points) Ded Moroz considers the function $f(x, y) = (1 + 3x + 2y)^{2022}$.
  
  Please, help him to find the second order Taylor approximation of this function at a point $x=0$, $y=0$.
  
  \newpage
\putyourname
\item (10 points) Consider the function 
\[
f(x, y) = \int_0^x 6e^{u^2} \, du  + \int_0^y 2\cos(u^2) \, du.   
\]

Find the gradient $\grad f$ at the point $(0, 0)$.


\newpage
\putyourname
  \item (10 points) Consider the system 
    \[
    \begin{cases}
      x + y + z = 2 \\
      2x^2 + 2y^2 = z^2 \\
    \end{cases}.  
    \]
    \begin{enumerate}
      \item Are the functions $x(z)$ and $y(z)$ defined in a neighborhood of the point $A(x=-1, y=1, z=2)$?
      \item Find $dy/dz$ at the point $A$ if possible. 
    \end{enumerate}
  
  
    \newpage
    \putyourname
      \item (10 points)  
  The set $S$ is defined by $S = \{(x, y) \in \mathbb{R}^2 \mid 0 \leq y \leq 5- x^2\}$. 
  Two rectangles one on the top of the other are inscribed in $S$, 
  thus they have the common side and the upper vertices lie on this parabola.
  Let $A_1 + A_2$ be the sum of their areas, where $A_1 >0$, $A_2 >0$. 
  
  Consider the maximization problem $A_1 + A_2 \to \max$.
  \begin{enumerate}
    \item Solve the maximization problem or show that the maximum does not exist.
    \item Check whether the Weierstrass theorem is applicable.
  \end{enumerate}
  
  \newpage
  \putyourname
    \item (10 points)  
  An implicit function is defined by equation $x^2 + y^2 - xy + 2x - 4y = 0$. Find all the point(s)
  $(x^*, y^*)$ on the curve represented by this equation, where
  \begin{enumerate}
    \item the tangent line to the curve is horizontal with the equation $y = c$;
    % \item this $c$ has the maximum possible value;
    \item conditions of the implicit function theorem are satisfied.
  \end{enumerate}
  
  \newpage
  \putyourname
    \item (10 points)  
  Consider the function $g(x) = \sqrt{x-4+\sqrt{x-4+\sqrt{x-4+\sqrt{\cdots}}}}$ defined for $x>4$.
  
  \begin{enumerate}
      \item Express $g^{2}(x)$ through the linear combination of $g(x)$ and $x-4$.
      \item Find $g'(5)$.
  \end{enumerate}
  
  
  %\item (10 points) The turtle starts at the point $(0, 0)$. 
  %Every day she continues her way at constant speed in constant direction. 
  %Every midnight she drops her speed by 60\% and turns $90^{\circ}$ counterclockwise. 
  %On the first day she goes by 1 kilometer in the positive direction of $OX$ axis.

  
  % What is the ultimate destination of the turtle?
  

  
  \end{enumerate}
  
  
  
  \newpage
  \chead{Section B}
  \putyourname  
  \begin{enumerate}[resume]
  
    \item (20 points) The Mean Value Theorem in Calculus claims that given a continuously differentiable function $f(x)$  
    on the closed segment $[x_0; x_0 + \Delta x]$, there exists $0 < \theta < 1$ such that $f(x_0 + \Delta x) - f(x_0) = 
    f'(x_0 + \theta \Delta x) \Delta x$. 
  
    Prove the version of this theorem for a function of the two variables $z = f(x,y) \in C^1$ in the following form: 
    there exists $0 < \theta < 1$ such that 
    \[
      f(x_0 + \Delta x, y_0 + \Delta y) - f(x_0, y_0) = \frac{\partial f}{\partial x}(x_0 + \theta \Delta x, y_0 + \theta \Delta y)\Delta x + \frac{\partial f}{\partial y}(x_0 + \theta \Delta x, y_0 + \theta \Delta y)\Delta y.  
    \]
    
    
    Hint. Join the 2 points $(x_0, y_0)$ and $(x_0 + \Delta x, y_0 + \Delta y)$ with the segment of a straight line passing through these points 
    and apply the chain rule to the composite function $f(x_0 + t \Delta x, y_0 + t \Delta y)$ where $t\in [0;1]$.
    
    
    \newpage
\putyourname
\item An economist solves a problem of the minimization of expenses of an individual whose utility function is $u(x, y) = \sqrt{x} + \sqrt{y}$. 
    Expenses are calculated by the formula $E=5x + 2y$ and the consumer would prefer to fix the value of the utility at 7 utiles.
    \begin{enumerate}
    \item (5 points) Formulate the problem of the expenses minimization and form a Lagrangian of this problem.
    \item (5 points) Using first-order conditions find the minimizing bundle $(x^*, y^*)$.
    \item (5 points) Using bordered Hessian or otherwise check the sufficiency condition.
    \item (5 points) A consumer decided to improve her welfare by adding additional 0,1 to 7 utiles. 
    How this decision will affect the expenditure value $E^* = 5x^* + 2y^*$, 
    where the bundle $(x^*, y^*)$ corresponds to a greater utility value? 
    Use appropriate Envelope Theorem to estimate $E^*$.
  \end{enumerate}
  
  \end{enumerate}
  
  




\end{document}
