\documentclass[12pt]{article} % размер шрифта

\usepackage{tikz} % картинки в tikz
\usepackage{microtype} % свешивание пунктуации

\usepackage{array} % для столбцов фиксированной ширины

\usepackage{url} % для вставки ссылок \url{...}

\usepackage{indentfirst} % отступ в первом параграфе

\usepackage{sectsty} % для центрирования названий частей
\allsectionsfont{\centering} % приказываем центрировать все sections

\usepackage{amsthm} % теоремы и доказательства

\theoremstyle{definition} % прямой шрифт в условии теорем
\newtheorem{theorem}{Теорема}[section]


\usepackage{amsmath, amssymb} % куча стандартных математических плюшек

\usepackage[top=2cm, left=1.5cm, right=1.5cm, bottom=2cm]{geometry} % размер текста на странице

\usepackage{lastpage} % чтобы узнать номер последней страницы

\usepackage{enumitem} % дополнительные плюшки для списков
%  например \begin{enumerate}[resume] позволяет продолжить нумерацию в новом списке
\usepackage{caption} % подписи к картинкам без плавающего окружения figure

\usepackage{ifthen}

\usepackage{fancyhdr} % весёлые колонтитулы
\pagestyle{fancy}
\lhead{Math for economists}
\chead{}
\rhead{2020-03-XX, ICEF}
\lfoot{Variant $\mu$}
\cfoot{Marking scheme}
\rfoot{Total time: 120 min}
\renewcommand{\headrulewidth}{0.4pt}
\renewcommand{\footrulewidth}{0.4pt}



\usepackage{todonotes} % для вставки в документ заметок о том, что осталось сделать
% \todo{Здесь надо коэффициенты исправить}
% \missingfigure{Здесь будет картина Последний день Помпеи}
% команда \listoftodos — печатает все поставленные \todo'шки

\usepackage{booktabs} % красивые таблицы
% заповеди из документации:
% 1. Не используйте вертикальные линии
% 2. Не используйте двойные линии
% 3. Единицы измерения помещайте в шапку таблицы
% 4. Не сокращайте .1 вместо 0.1
% 5. Повторяющееся значение повторяйте, а не говорите "то же"

\usepackage{fontspec} % поддержка разных шрифтов
\usepackage{polyglossia} % поддержка разных языков

\setmainlanguage{english}
\setotherlanguages{russian}

\setmainfont{Linux Libertine O} % выбираем шрифт
% если Linux Libertine не установлен, то
% можно также попробовать Helvetica, Arial, Cambria и т.Д.

% чтобы использовать шрифт Linux Libertine на личном компе,
% его надо предварительно скачать по ссылке
% http://www.linuxlibertine.org/index.php?id=91&L=1

% на сервисах типа sharelatex.com этот шрифт есть :)

\newfontfamily{\cyrillicfonttt}{Linux Libertine O}
% пояснение зачем нужно шаманство с \newfontfamily
% http://tex.stackexchange.com/questions/91507/

\AddEnumerateCounter{\asbuk}{\russian@alph}{щ} % для списков с русскими буквами
%\setlist[enumerate, 2]{label=\asbuk\cdot),ref=\asbuk\cdot} % списки уровня 2 будут буквами а) б) ...

%% эконометрические и вероятностные сокращения
\DeclareMathOperator{\Cov}{Cov}
\DeclareMathOperator{\Corr}{Corr}
\DeclareMathOperator{\Var}{Var}
\DeclareMathOperator{\E}{E}
\DeclareMathOperator{\grad}{grad}
\def \hb{\hat{\beta}}
\def \hs{\hat{\sigma}}
\def \htheta{\hat{\theta}}
\def \s{\sigma}
\def \hy{\hat{y}}
\def \hY{\hat{Y}}
\def \v1{\vec{1}}
\def \e{\varepsilon}
\def \he{\hat{\e}}
\def \z{z}
\def \hVar{\widehat{\Var}}
\def \hCorr{\widehat{\Corr}}
\def \hCov{\widehat{\Cov}}
\def \cN{\mathcal{N}}

\def \putyourname{\fbox{
    \begin{minipage}{42em}
      Name, group no:\vspace*{3ex}\par
      \noindent\dotfill\vspace{2mm}
    \end{minipage}
  }
}



\begin{document}



Brief solutions and marking scheme for March demo mock.



\section*{Section A}


\begin{enumerate}

\item Five points will be given for the correct equation representing the set in $(x, y)$-plane. 
Completing perfect squares will provide another 2 points. 
Drawing of this circle with labeling its center with exposition that the origin belongs to it 
(that is all we need from sketchy drawing) will give final 3 points.

\item When we multiply this equation by $xyz$ and divide by 2 we get 
familiar Euler’s equation for a function $f(x, y, z)$. 
That is necessary and sufficient condition for a function to be homogeneous and 
we can establish its degree of homogeneity which is in this case $0.5$. 
This correctly stated will give 5 points. 
Another theoretical fact that first-order derivative of a homogeneous function 
with respect to any variable brings down its degree by 1 will give us the final answer $-0.5$. 
This reference to theory along with correct answer will give another 5 points.

\item to be inserted later

\item to be inserted later

\item This is a maximization problem with one equality constraint. 
We shall solve it by Lagrange method. 
Firstly we check NDCQ condition: 
the gradient of constraint $(2x, 2y, 2z) = (0, 0, 0)$ is zero at the origin 
but the origin does not belong to constraint set. 
Proper NDCQ trial gives 2 points. 
Lagrangian forming with correctly written FOCs gives 3 points. 
The substitution of $(x, y, z) = (5/2\lambda, 1/\lambda, 3/2\lambda)$ 
into constraint equation gives two values to $\lambda$: $\sqrt{19/12}$ and $-\sqrt{19/12}$. 
This correct part of solution gives another 2 points. 
It would be reasonable at this stage to apply Weierstrass’ theorem. 
This theorem is readily applicable since 
the objective function is continuous and the constraint set is compact. 
Both maximum and minimum values globally are attained 
(this theorem is about global properties). 
For a maximum we need to choose positive $\lambda$. 
This reference to theory and correct answer will provide extra 3 points. 
Answer: $(x, y, z) = (5, 2, 3)$, optimal value is $\sqrt{3/19}$.

\item Characteristic equation for Fibonacci’s equation is $q^2 - q -1$ 
with the roots $(1+\sqrt 5)/2$, $(1-\sqrt 5)/2$.  
These findings will give 2 points. 
The formula for general solution $F_t = C_1 q_1^t + C_2 q_2^t$ gives another 3 points. 
Set the system for $C_1$ and $C_2$  and solve it: 
\[
\begin{cases}
    C_1 + C_2 = -1 \\
    C_1 q_1 + C_2 q_2 = 4 \\
\end{cases}
\]

Then $C_1 = (9 - \sqrt 5)/10$, $C_2 = (-9 - \sqrt 5)/10$. 
Here 1 point is awarded for correct system and another 3 points for correct coefficients values. 
After substitution of these values into general solution we get final answer (extra 1 point). 
This answer does not require further simplification.
\end{enumerate}


\section*{Section B}

\begin{enumerate}[resume]

\item We start with NDCQ checking (worth 3 points). 
Jacobian matrix based on these constraints is:
\[
J = \begin{pmatrix}
    1 & 1 & 1 \\
    1 & 1 & 1.5 \\
\end{pmatrix}.
\]

This matrix has non-zero rows and matrix has itself has the maximal rank which is 2. 
After forming Lagrangian (3 points) 
\[
L = 2x^2 + 4y^2 +xy + 8z^2 + 2yz + \lambda (1 - x - y - z) + \mu (5/6 - x - y - 3z/2)  
\]
we derive FOCs:   
\[
\begin{cases}
4x + y - \lambda - \mu = 0 \\
8y + x + 2z -\lambda - \mu = 0 \\
16z + 2y - \lambda - 3\mu/2 = 0 \\
\lambda, \mu \geq 0 \\
\lambda (1 - x - y - z) =0, \quad \mu (5/6 - x - y - 3z/2) = 0 \\
1 - x - y - z \geq 0, \quad 5/6 - x - y - 3z/2 \geq 0
\end{cases}
\]

The solution of any such system of conditions is based on considering all possible cases. 
The problem is in identifying possible shortcuts that reduce the work. 
Let us at first assume that $x+y+z >1$ (one of the cases to be considered). 
Duly considered it will give 5 points. 
Then either $5/6 - x - y - 3z/2 >0$  or $5/6 - x - y - 3z/2 =0$. 


In the case $5/6 - x - y - 3z/2 >0$ since $\lambda = \mu =0$ (guess why?) 
the FOC system easily provides $x=y=z=0$ which does not satisfy constraint set. 


In $5/6 - x - y - 3z/2 =0$ we deal with a system in 4 equations with 4 unknowns. 
This system should be reduced by Gaussian elimination procedure up to a point when we are able to assess the sign of $z$ alone. 
Not hard to see  and this violates $x+y+z>1$ condition. 


Then we proceed with $x+y+z =1$ case.
Here $z = 1 - x- y$. 
After substituting into objective function we get
\[
f(x, y) = 10x^2  + 10y^2 + 15xy -16x -14y + 8
\]
under constraint $x+y \leq 3/5$. 
Here again we consider two cases: binding and nonbinding. 
If constraint does not bind then we get from FOC $x=22/35$, $y=1/70$. 
It violates constraint. 
A binding case gives another point, $x=13/15$, $y=7/15$. 
Here Weierstrass theorem can not be applied (why?). 
We may 
check bordered Hessian instead:
\[
\det
\begin{pmatrix}
    0 & 1 & 1\\
    1 & 20 & 15 \\
    1 & 15 & 20 \\
\end{pmatrix} = -10
\] 

We conclude that this is the local minimum. 
If you have successfully reached that point you deserved 20 points. 


Globality exposition of this minimum will give some extra points. 
The simplest proof is as follows: we are solving a minimization problem 
for an objective function which is a positively defined quadratic form. 
The constraint set in 3d is a straight line (a line of intersection of the 2 planes). 
On the points of this line this form becomes just a quadratic function facing up. 
Then the point of minimum is global.

\item Firstly we solve characteristic equation 
\[
  q^4 + 7q^2 -18q +10=0  
\] 
Writing down this equation is worth 1 point. 
Among divisors of number 10 we choose the smallest which is 1 and substitute: $q_1 =1$. 
After dividing polynomial by $q-1$ we get the quotient and the equation:
\[
  q^3 +q^2 + 8q -10=0  
\]
We try again 1 and find $q_2 = 1$. 
The resulting quadratic equation has two complex roots $q_3 = -1 -3i$, $q_4 = -1+3i$. 
Correct finding of all the roots provide 6 points. 


Now  we can write down complementary function:
\[
y_{cf}(x) = C_1 e^x + C_2 x e^x + e^{-x}(C_3 \cos 3x + C_4 \sin 3x)
\]

We get for this step another 3 points. 
Now we seek a particular integral as a sum of two terms. 

The first term results from $x$ in the right side (no resonance here). 
Undetermined coefficients method gives us $y_1 = 0.1x + 0.18$ (worth 3 points). 

Finding the second term of particular integral is much more complicated (resonance case). 
Performed correctly it will bring 6 points. So we seek this term in the form of $y_2 =ax^2 e^x$.
 The ultimate 1 point is given when the solution is written correctly. 
It is useful to use Leibnitz formula for the multiple differentiation of the product. 
Finally we get $y_2 = x^2 e^x /26$. And the final answer is:
\[
y(x) = C_1 e^x + C_2 x e^x + e^{-x}(C_3 \cos 3x + C_4 \sin 3x) + 0.1x + 0.18 +  x^2 e^x /26,
\]
with all $C_i \in \mathbb{R}$.

A remark on computational mistakes: 
if the route for solution is correct with the understanding of all performed operations 
then the range for points deduction is from 1 to 5 points 
depending on severity of consequences resulted from these mistakes.
\end{enumerate}

\end{document}