\documentclass[12pt,a4paper]{article}
\usepackage[utf8]{inputenc}
\usepackage[russian]{babel}

\usepackage{amsmath}
\usepackage{amsfonts}
\usepackage{amssymb}
\usepackage[left=2cm,right=2cm,top=2cm,bottom=2cm]{geometry}

\begin{document}
\section*{MOR. Retake --- 07.09.2015}

\pagestyle{empty}

\begin{enumerate}
\item Using Lagrange multipliers maximize  the function $f(x_1,x_2,x_3)=-x_1-2x_2+3x_3$ subject to constraints:   $2(x_1+1)^2+x_2^2+3(x_3-1)^2\leq 5$ and $x_1, x_2, x_3 \geq 0$. Find the point(s) of maximum and maximum value of $f$. Justify your answer by reference to Weierstrass theorem if it is relevant or otherwise. Carefully state any theorem you use.

\item For all real values of parameter $\beta$ that lies within the range $-1<\beta<0$ maximize linear function $2x_1-x_2+8x_3-19$ subject to constraints  $x_1\geq x_2+x_3+\beta$,  $2x_1+x_2+4x_3\leq \beta+1$ and $x_1, x_2, x_3 \geq 0$. You are not asked to find the maximizer.

\item Find the general solution of the differential equation $y''+6y'+9y=xe^{-2x}+\cos(x)$
\item Consider the system of difference equations
$$ 
\left\{ \begin{array}{l}
x_{t+1}=2x_t-4y_t \\
y_{t+1}=x_t-3y_t+3
\end{array} \right.
$$
\begin{enumerate}
\item Solve the system
\item Find the equilibrium solution and check whether it's stable
\end{enumerate}


\item Find all pure and mixed Nash equilibria in the following bimatrix game:


\begin{tabular}{c|ccc}
 & d & e & f \\ 
\hline 
a & 4;5 & 1;4 & 1;1  \\ 
b & 2;8 & 5;0 & 0;4  \\ 
c & 0;3 & 2;2 & 5;7  \\ 
\end{tabular} 


\item There is an auction of a painting with two players. The value of the painting for the first player is a random variable $v_1$, for the second player --- $v_2$. The random variables $v_1$ and $v_2$ are independent and identically distributed from 0 to 1 million dollars with density function $f(t)=2t$. Each player makes the bid $b_i$ knowing only his own value of the painting. The player who makes the highest bid gets the painting and pays his bid. 

Find a Nash equilibrium where each player uses linear strategy of the form $b_i=k\cdot v_i$.


\end{enumerate}



\end{document}