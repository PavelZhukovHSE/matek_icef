\documentclass[12pt,a4paper]{article}
\usepackage[utf8]{inputenc}
\usepackage[english]{babel}
\usepackage{amsmath}
\usepackage{amsfonts}
\usepackage{amssymb}
\usepackage{enumitem}
\usepackage[left=1cm,right=1cm,top=1cm,bottom=1cm]{geometry}

\def \RR{\mathbb{R}}

\begin{document}
\thispagestyle{empty}
\textbf{Variant 1}. Please, don't forget to write you variant number. Sections A and B will make up 60\% and 40\% of the exam grade, respectively. Total duration of the exam is 120 min. Good luck! :) 



\textbf{SECTION A}

\begin{enumerate}

\item Consider the function $f(x,y,z)=x^5+2xyz-3z^3$. Using the total differential find the approximate value of $f(1.01,0.99,1.01)$.

\item Consider the function $f(x,y)=2x^4-(x+y)^3$. 
\begin{enumerate}
\item Find the Hesse matrix. Clearly state the Young theorem even if you don't use it.
\item Find the definiteness (positive definite, positive semidefinite, etc) of the Hesse matrix at the point $(1,2)$.
\end{enumerate}

\item Let the function $f(x,y)$ be defined by the formula
\[
f(x,y)=\begin{cases}
-1, \, \text{if} \, x>y \\
1, \, \text{if} \, x\leq y
\end{cases}
\]

\begin{enumerate}
\item Find the limits $\lim_{x\to\infty}\lim_{y\to \infty} f(x,y)$ and $\lim_{y\to\infty}\lim_{x\to \infty} f(x,y)$
\item Does the limit $\lim_{x\to\infty, \, y\to \infty} f(x,y)$ exist?
\end{enumerate}


\item The functions $f$ and $g$ are given: $f(x,y)=x^2+2xy+y^4$, $g(x,y)=-5x^2-xy-2y^4$. Find at least one direction from the point $(1,1)$ in which both functions will grow.

\item The function $z$ is defined by the formula $z(x,y)=f(x^3-y^2)$. Simplify the expression $2y\frac{\partial z}{\partial x}+3x^2\frac{\partial z}{\partial y}$. 

\item Consider the function $f(x,y)=\sqrt{x+\sqrt{y+\sqrt{x+\sqrt{y + \ldots }}}}$. 
\begin{enumerate}
\item Find the value of $(f^2(x,y)-x)^2-y-f(x,y)$
\item Find $\partial f/\partial x$ and $\partial f/\partial y$ at the point $(1,1)$
\end{enumerate}

 


%\item Given the system
%\begin{equation} \nonumber
%\begin{cases}
%u^2-w^2+x^2+y^2=2 \\
%uw+xy=2
%\end{cases}
%\end{equation}
%\begin{enumerate}
%\item Define a sufficient condition for functions $u(x,y)$ and $w(x,y)$ to be differentiable
%\item Find $\frac{\partial u}{\partial x}$
%\end{enumerate}

\end{enumerate}

\textbf{SECTION B}

\begin{enumerate}[resume]


\item (20 points) Let $S_1$ and $S_2$ be two sets from $\RR^2$: $S_1=\{(x,y) \in \RR^2 | xy=1 \}$, $S_2=\{(x,y) \in \RR^2 | xy=-1 \}$ and $S=S_1+S_2$. We denote by the sum $S_1+S_2$ the set 
\[
S_1+S_2=\{ (x,y) \in \RR^2 | (x,y)=(x_1,y_1)+(x_2,y_2), \, (x_1,y_1) \in S_1, \, (x_2,y_2) \in S_2 \}
\]
\begin{enumerate}
\item Are the sets $S_1$ and $S_2$ closed? Justify your answer.
\item Does the origin belongs to the set $S$? % Show that the origin does not belong to $S$.
\item Is the set $S$ closed? % Show that $S$ is not closed.
\end{enumerate}
\item (20 points) Consider a Cournot duopoly of the two identical firms that compete by choosing outputs $y_1$  and $y_2$  simultaneously. Marginal costs of these firms are constant $MC_1=MC_2=c>0$. When the outputs $y_1$  and $y_2$ are set, the price of a good can be found by the formula  $p=a-b(y_1+y_2)$, where  $a>c$, $b>0$.
\begin{enumerate}
\item Find equations of the level curves for the profits of the firms $\pi_1(y_1, y_2)$ and  $\pi_2(y_1, y_2)$.
\item It is known that the point of equilibrium outputs $(y_1^*, y_2^*)$  in the coordinate plane $(y_1, y_2)$ can be found by drawing tangent lines to the level curves and these tangents should be parallel to the axes. 
Then $(y_1^*, y_2^*)$ is the point of intersection of the tangents. 
By finding corresponding gradients of $\pi_1(y_1, y_2)$,  $\pi_2(y_1, y_2)$ and using the hint stated above, find    $(y_1^*, y_2^*)$ in terms of $a$, $b$ and $c$.
\end{enumerate}

\end{enumerate}



\newpage
\thispagestyle{empty}
\textbf{Variant 2}. Please, don't forget to write you variant number. Sections A and B will make up 60\% and 40\% of the exam grade, respectively. Total duration of the exam is 120 min. Good luck! :) 



\textbf{SECTION A}

\begin{enumerate}

\item Consider the function $f(x,y,z)=2x^5+2xyz-3z^3$. Using the total differential find the approximate value of $f(1.02,0.98,1.01)$.

\item Consider the function $f(x,y)=3x^4-(x+y)^3$. 
\begin{enumerate}
\item Find the Hesse matrix. Clearly state the Young theorem even if you don't use it.
\item Find the definiteness (positive definite, positive semidefinite, etc) of the Hesse matrix at the point $(1,3)$.
\end{enumerate}

\item Let the function $f(x,y)$ be defined by the formula
\[
f(x,y)=\begin{cases}
-2, \, \text{if} \, x>y \\
2, \, \text{if} \, x\leq y
\end{cases}
\]

\begin{enumerate}
\item Find the limits $\lim_{x\to\infty}\lim_{y\to \infty} f(x,y)$ and $\lim_{y\to\infty}\lim_{x\to \infty} f(x,y)$
\item Does the limit $\lim_{x\to\infty, \, y\to \infty} f(x,y)$ exist?
\end{enumerate}


\item The functions $f$ and $g$ are given: $f(x,y)=x^2+2xy+y^4$, $g(x,y)=-6x^2-xy-2y^4$. Find at least one direction from the point $(1,1)$ in which both functions will grow.

\item The function $z$ is defined by the formula $z(x,y)=f(x^3-2y^3)$. Simplify the expression $6y^2\frac{\partial z}{\partial x}+3x^2\frac{\partial z}{\partial y}$. 

\item Consider the function $f(x,y)=\sqrt{x+\sqrt{2y+\sqrt{x+\sqrt{2y + \ldots }}}}$. 
\begin{enumerate}
\item Find the value of $(f^2(x,y)-x)^2-2y-f(x,y)$
\item Find $\partial f/\partial x$ and $\partial f/\partial y$ at the point $(1,1)$
\end{enumerate}

 


%\item Given the system
%\begin{equation} \nonumber
%\begin{cases}
%u^2-w^2+x^2+y^2=2 \\
%uw+xy=2
%\end{cases}
%\end{equation}
%\begin{enumerate}
%\item Define a sufficient condition for functions $u(x,y)$ and $w(x,y)$ to be differentiable
%\item Find $\frac{\partial u}{\partial x}$
%\end{enumerate}

\end{enumerate}

\textbf{SECTION B}

\begin{enumerate}[resume]


\item (20 points) Let $S_1$ and $S_2$ be two sets from $\RR^2$: $S_1=\{(x,y) \in \RR^2 | xy=1 \}$, $S_2=\{(x,y) \in \RR^2 | xy=-1 \}$ and $S=S_1+S_2$. We denote by the sum $S_1+S_2$ the set 
\[
S_1+S_2=\{ (x,y) \in \RR^2 | (x,y)=(x_1,y_1)+(x_2,y_2), \, (x_1,y_1) \in S_1, \, (x_2,y_2) \in S_2 \}
\]
\begin{enumerate}
\item Are the sets $S_1$ and $S_2$ closed? Justify your answer.
\item Does the origin belongs to the set $S$? % Show that the origin does not belong to $S$.
\item Is the set $S$ closed? % Show that $S$ is not closed.
\end{enumerate}
\item (20 points) Consider a Cournot duopoly of the two identical firms that compete by choosing outputs $y_1$  and $y_2$  simultaneously. Marginal costs of these firms are constant $MC_1=MC_2=c>0$. When the outputs $y_1$  and $y_2$ are set, the price of a good can be found by the formula  $p=a-b(y_1+y_2)$, where  $a>c$, $b>0$.
\begin{enumerate}
\item Find equations of the level curves for the profits of the firms $\pi_1(y_1, y_2)$ and  $\pi_2(y_1, y_2)$.
\item It is known that the point of equilibrium outputs $(y_1^*, y_2^*)$  in the coordinate plane $(y_1, y_2)$ can be found by drawing tangent lines to the level curves and these tangents should be parallel to the axes. 
Then $(y_1^*, y_2^*)$ is the point of intersection of the tangents. 
By finding corresponding gradients of $\pi_1(y_1, y_2)$,  $\pi_2(y_1, y_2)$ and using the hint stated above, find    $(y_1^*, y_2^*)$ in terms of $a$, $b$ and $c$.
\end{enumerate}

\end{enumerate}

\newpage
\thispagestyle{empty}
\textbf{Variant 3}. Please, don't forget to write you variant number. Sections A and B will make up 60\% and 40\% of the exam grade, respectively. Total duration of the exam is 120 min. Good luck! :) 



\textbf{SECTION A}

\begin{enumerate}

\item Consider the function $f(x,y,z)=x^5+3xyz-2z^3$. Using the total differential find the approximate value of $f(1.03,0.99,1.01)$.

\item Consider the function $f(x,y)=2x^4-2(x+y)^3$. 
\begin{enumerate}
\item Find the Hesse matrix. Clearly state the Young theorem even if you don't use it.
\item Find the definiteness (positive definite, positive semidefinite, etc) of the Hesse matrix at the point $(2,2)$.
\end{enumerate}

\item Let the function $f(x,y)$ be defined by the formula
\[
f(x,y)=\begin{cases}
-3, \, \text{if} \, x>y \\
3, \, \text{if} \, x\leq y
\end{cases}
\]

\begin{enumerate}
\item Find the limits $\lim_{x\to\infty}\lim_{y\to \infty} f(x,y)$ and $\lim_{y\to\infty}\lim_{x\to \infty} f(x,y)$
\item Does the limit $\lim_{x\to\infty, \, y\to \infty} f(x,y)$ exist?
\end{enumerate}


\item The functions $f$ and $g$ are given: $f(x,y)=x^2+2xy+y^4$, $g(x,y)=-7x^2-xy-2y^4$. Find at least one direction from the point $(1,1)$ in which both functions will grow.

\item The function $z$ is defined by the formula $z(x,y)=f(x^3-y^5)$. Simplify the expression $5y^4\frac{\partial z}{\partial x}+3x^2\frac{\partial z}{\partial y}$. 

\item Consider the function $f(x,y)=\sqrt{x+\sqrt{3y+\sqrt{x+\sqrt{3y + \ldots }}}}$. 
\begin{enumerate}
\item Find the value of $(f^2(x,y)-x)^2-3y-f(x,y)$
\item Find $\partial f/\partial x$ and $\partial f/\partial y$ at the point $(1,1)$
\end{enumerate}

 


%\item Given the system
%\begin{equation} \nonumber
%\begin{cases}
%u^2-w^2+x^2+y^2=2 \\
%uw+xy=2
%\end{cases}
%\end{equation}
%\begin{enumerate}
%\item Define a sufficient condition for functions $u(x,y)$ and $w(x,y)$ to be differentiable
%\item Find $\frac{\partial u}{\partial x}$
%\end{enumerate}

\end{enumerate}

\textbf{SECTION B}

\begin{enumerate}[resume]


\item (20 points) Let $S_1$ and $S_2$ be two sets from $\RR^2$: $S_1=\{(x,y) \in \RR^2 | xy=1 \}$, $S_2=\{(x,y) \in \RR^2 | xy=-1 \}$ and $S=S_1+S_2$. We denote by the sum $S_1+S_2$ the set 
\[
S_1+S_2=\{ (x,y) \in \RR^2 | (x,y)=(x_1,y_1)+(x_2,y_2), \, (x_1,y_1) \in S_1, \, (x_2,y_2) \in S_2 \}
\]
\begin{enumerate}
\item Are the sets $S_1$ and $S_2$ closed? Justify your answer.
\item Does the origin belongs to the set $S$? % Show that the origin does not belong to $S$.
\item Is the set $S$ closed? % Show that $S$ is not closed.
\end{enumerate}
\item (20 points) Consider a Cournot duopoly of the two identical firms that compete by choosing outputs $y_1$  and $y_2$  simultaneously. Marginal costs of these firms are constant $MC_1=MC_2=c>0$. When the outputs $y_1$  and $y_2$ are set, the price of a good can be found by the formula  $p=a-b(y_1+y_2)$, where  $a>c$, $b>0$.
\begin{enumerate}
\item Find equations of the level curves for the profits of the firms $\pi_1(y_1, y_2)$ and  $\pi_2(y_1, y_2)$.
\item It is known that the point of equilibrium outputs $(y_1^*, y_2^*)$  in the coordinate plane $(y_1, y_2)$ can be found by drawing tangent lines to the level curves and these tangents should be parallel to the axes. 
Then $(y_1^*, y_2^*)$ is the point of intersection of the tangents. 
By finding corresponding gradients of $\pi_1(y_1, y_2)$,  $\pi_2(y_1, y_2)$ and using the hint stated above, find    $(y_1^*, y_2^*)$ in terms of $a$, $b$ and $c$.
\end{enumerate}

\end{enumerate}



\newpage
\thispagestyle{empty}
\textbf{Variant 4}. Please, don't forget to write you variant number. Sections A and B will make up 60\% and 40\% of the exam grade, respectively. Total duration of the exam is 120 min. Good luck! :) 



\textbf{SECTION A}

\begin{enumerate}

\item Consider the function $f(x,y,z)=2x^5+2xyz-2z^3$. Using the total differential find the approximate value of $f(1.04,0.99,1.01)$.

\item Consider the function $f(x,y)=3x^4-(x+y)^3$. 
\begin{enumerate}
\item Find the Hesse matrix. Clearly state the Young theorem even if you don't use it.
\item Find the definiteness (positive definite, positive semidefinite, etc) of the Hesse matrix at the point $(2,1)$.
\end{enumerate}

\item Let the function $f(x,y)$ be defined by the formula
\[
f(x,y)=\begin{cases}
-4, \, \text{if} \, x>y \\
4, \, \text{if} \, x\leq y
\end{cases}
\]

\begin{enumerate}
\item Find the limits $\lim_{x\to\infty}\lim_{y\to \infty} f(x,y)$ and $\lim_{y\to\infty}\lim_{x\to \infty} f(x,y)$
\item Does the limit $\lim_{x\to\infty, \, y\to \infty} f(x,y)$ exist?
\end{enumerate}


\item The functions $f$ and $g$ are given: $f(x,y)=x^2+2xy+y^4$, $g(x,y)=-8x^2-xy-2y^4$. Find at least one direction from the point $(1,1)$ in which both functions will grow.

\item The function $z$ is defined by the formula $z(x,y)=f(x^4-y^2)$. Simplify the expression $2y\frac{\partial z}{\partial x}+4x^3\frac{\partial z}{\partial y}$. 

\item Consider the function $f(x,y)=\sqrt{x+\sqrt{4y+\sqrt{x+\sqrt{4y + \ldots }}}}$. 
\begin{enumerate}
\item Find the value of $(f^2(x,y)-x)^2-4y-f(x,y)$
\item Find $\partial f/\partial x$ and $\partial f/\partial y$ at the point $(1,1)$
\end{enumerate}

 


%\item Given the system
%\begin{equation} \nonumber
%\begin{cases}
%u^2-w^2+x^2+y^2=2 \\
%uw+xy=2
%\end{cases}
%\end{equation}
%\begin{enumerate}
%\item Define a sufficient condition for functions $u(x,y)$ and $w(x,y)$ to be differentiable
%\item Find $\frac{\partial u}{\partial x}$
%\end{enumerate}

\end{enumerate}

\textbf{SECTION B}

\begin{enumerate}[resume]


\item (20 points) Let $S_1$ and $S_2$ be two sets from $\RR^2$: $S_1=\{(x,y) \in \RR^2 | xy=1 \}$, $S_2=\{(x,y) \in \RR^2 | xy=-1 \}$ and $S=S_1+S_2$. We denote by the sum $S_1+S_2$ the set 
\[
S_1+S_2=\{ (x,y) \in \RR^2 | (x,y)=(x_1,y_1)+(x_2,y_2), \, (x_1,y_1) \in S_1, \, (x_2,y_2) \in S_2 \}
\]
\begin{enumerate}
\item Are the sets $S_1$ and $S_2$ closed? Justify your answer.
\item Does the origin belongs to the set $S$? % Show that the origin does not belong to $S$.
\item Is the set $S$ closed? % Show that $S$ is not closed.
\end{enumerate}
\item (20 points) Consider a Cournot duopoly of the two identical firms that compete by choosing outputs $y_1$  and $y_2$  simultaneously. Marginal costs of these firms are constant $MC_1=MC_2=c>0$. When the outputs $y_1$  and $y_2$ are set, the price of a good can be found by the formula  $p=a-b(y_1+y_2)$, where  $a>c$, $b>0$.
\begin{enumerate}
\item Find equations of the level curves for the profits of the firms $\pi_1(y_1, y_2)$ and  $\pi_2(y_1, y_2)$.
\item It is known that the point of equilibrium outputs $(y_1^*, y_2^*)$  in the coordinate plane $(y_1, y_2)$ can be found by drawing tangent lines to the level curves and these tangents should be parallel to the axes. 
Then $(y_1^*, y_2^*)$ is the point of intersection of the tangents. 
By finding corresponding gradients of $\pi_1(y_1, y_2)$,  $\pi_2(y_1, y_2)$ and using the hint stated above, find    $(y_1^*, y_2^*)$ in terms of $a$, $b$ and $c$.
\end{enumerate}

\end{enumerate}


\end{document}