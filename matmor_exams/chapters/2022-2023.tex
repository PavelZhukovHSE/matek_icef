\subsection{MFE, december exam, 2022-12-27}

Short rules: offline exam, 2 hours, closed book. 
\begin{enumerate}
      \item (10 points) Consider the sets $B_n=\left[ \frac{1}{3n},\frac{2n-1}{n}\right]\subset \mathbb{R}$. Let $A=\bigcup \limits_{n=1}^{\infty}B_n$.  
        \begin{enumerate}
            \item Is $A$ bounded? open? closed? compact?
            \item Sketch $A\times A$. Is it open? closed?
        \end{enumerate} 



    \item (10 points) Find the limit or prove it doesn't exist $$\lim\limits_{x\to +\infty, y\to +\infty}x^2y^5\sin {\frac{1}{x^5y^2}}.$$



    \item (10 points) Using the chain rule calculate all partial derivatives of the first and second order of $g(\alpha,\beta)=f(x(\alpha,\beta),y(\alpha,\beta))$, if $x=2\alpha-\beta^2, y=\alpha \beta$ and $f(x,y)$ is twice continuously differentiable. 



    \item (10 points) Two dimensional function is given by $f(x,y)=x^3-2x^2y$.
        \begin{enumerate} 
            \item Find the direction of maximal growth of $f(x,y)$ from the point $A(1,1)$.
            \item Find at least one direction from the point $A$ in which the function doesn't increase and doesn't decrease (doesn't change).
        \end{enumerate} 


    \item (10 points) Find and classify all the critical points of  $f(x,y)=(y^2-2xy+x)e^x.$


    \item (10 points) Using Lagrange multiplier method find and classify the constrained extrema of $f(x,y,z)=3x-y+2z$ subject to $4x^2+4y^2+z^2=13$.
\end{enumerate}

\chead{Section B}

\begin{enumerate}[resume]
    \item (20 points) Let $h: \mathbb{R} \to \mathbb{R}$ be a $\mathbb{C}^1$ function and $h(0)=0$. Consider the system

$$
    \begin{cases}
      e^x+h(y)=u^2\\
      e^y-h(x)=v^2
    \end{cases}   
$$
            \begin{enumerate} 
                \item Prove that in a neighbourhood of $(x,y,u,v)=(0,0,1,1)$ we can define $x(u,v)$ and $y(u,v)$ as functions of $u$ and $v$.
                \item If it is also known that $h'(0)=1$ calculate $\frac{\partial ^2 x}{\partial v^2}$ at the point $(0,0,1,1)$.
            \end{enumerate}
\end{enumerate}

\begin{enumerate}[resume]
        \item  In a two-product economy the price-taking industry bears total costs $C(x,y)=c_1(x)+c_2(y)$, where $x$  and $y$  are the outputs and   and   are the functions twice continuously differentiable with the positive first and second order derivatives. Since only one factor of production is employed, the outputs are related $2x+3y=L$ , where $L$  is the total endowment of labor.
            \begin{enumerate}
                \item (5 points) Set the minimization problem for $C(x,y)=c_1(x)+c_2(y)$, subject to constraint and write down first-order conditions applied to the Lagrangian of the problem $K(x,y,\lambda)$.
	           \item (5 points) By checking the bordered Hessian determinant, prove that the only critical point of $K(x,y,\lambda)$ is a minimizer.
		        \item(3 points) Let this point be $(x^*,y^*,\lambda^*)$. Find $\frac{dC(x^*,y^*)}{dL}$. Use one of the envelope theorems.
		        \item(7 points) Industry economists have found out that $c_1''(x^*)=c_2''(y^*)$. Use this information to calculate the values of $\frac{dx^*}{dL}$ and $\frac{dy^*}{dL}$.
            \end{enumerate}
    \end{enumerate}

