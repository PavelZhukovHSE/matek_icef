% !TEX root = ../matmor_exams.tex

\subsection{MFE, October mock, 2019-10-25}

The october exam was shorter than usual as it fell on the same day with linear algebra exam. 

\begin{enumerate}
    \item (15 points) Consider the function $f(x, y) = x^3 -3 y^3 + 2xy$.
    Using the total differential find the approximate value of $f(1.98, 0.99)$.
  

    \item (15 points) Consider the system
    \[
    \begin{cases}
    3x^3 + y^3 + z^2 = 5 \\
    x + x^3 + 2y^3x = 4 \\
    \end{cases}
    \]
  
    \begin{enumerate}
      \item Check whether the functions $z(y)$ and $x(y)$ are defined at a point $(1, 1, 1)$;
      \item Find $z'(y)$ if possible.
   \end{enumerate}
  

   \item (15 points) Consider the function $h(b) = f(f(f(b \cdot f(b))))$.
   Find $dh/db$ for $b=1$ if it is known that $f(1)=2$, $f(2)=3$, $f(3)=1$, 
   $f'(1)=3$, $f'(3)=2$, $f'(2)=1$.
  

  \item (15 points) Consider the function $f (x, y) = xyz^3$, the vector $v = (1, 2)$ and the point $A = (−1, −1)$.
  
  \begin{enumerate}
    \item  Find the gradient of $f$ at the point $A$.
  \item  Find the directional derivative of $f$ at the point $A$ in the direction given by $v$.
  \end{enumerate}
  
  \item (15 points) Provide an explicit example of a sequence in $\RR^2$ that is unbounded and has 
  exactly two accumulation points. 
  
  
  \item Two identical firms compete in a labor market with the supply function $w(L)= w_0 + aL$, 
  where $w_0>0$, $a>0$ and $L$ is the labor amount supplied at the wage rate $w$.
  
  In order to find equilibrium one has to solve the system of equations
  \[
  \begin{cases}
    f(L_1) - ME_1 = 0 \\
    f(L_2) - ME_2 = 0
  \end{cases},
  \]
  where $f'(L)<0$ for all $L>0$ 
  and $ME_1$, $ME_2$ are marginal expenses which are found by differentiation, 
  $ME_i = \partial (w(L)L_i)/\partial L_i$ for $i \in \{1, 2\}$ and $L = L_1 + L_2$. 
  
  Suppose the equilibrium exists.
  \begin{enumerate}
    \item (10 points) Prove that $L_1^*=L_2^*$.
    \item (15 points) Find $\partial L_1^*/\partial w_0$.
  \end{enumerate}
  
  \end{enumerate}
  

\subsection{MFE, October mock, 2019-10-25, marking}


\begin{enumerate}
  \item 1 - value of function at good point,
  2 - value of der-ve wrt x,
  2 - value of der-ve wrt y,
  4 - 1st Taylor series approximation,
  1 - final result
  
  \item 
  a) 1pt - functions are continuous, 3pts - Jacobean matrix, 1pt - point satisfies the system of eq-ns
  
  
  b) 2pts - Formula, 3pts - Final result
  
  \item 3pts chain rule, 2pts product rule, 2 pts each of 4 multiples, 2pts answer
  
  \item
  a) 2pts for partial derivatives, 2pts for derivatives in a point, 3 pts correct gradient
  
  b) 1pt correct formula, 3pts correct grad and direction vector, 2 pts normalization, 2 pts answer
  
  \item  any sequence in $R^2$ = +3 pts,
  unbounded = +6 pts,
  with two accumulation points = +6 pts

  \item
  \begin{enumerate}
\item correct $\partial w(L)L_i /\partial L_i$ = 4 pts, rest of proof = 6 pts
 \item   ratio of determinants formula = 5 pts, nominator = 5 pts, denominator = 5 pts
  
\end{enumerate}
  

\end{enumerate}
