% !TEX root = ../matmor_exams.tex

\subsection{MFE, december exam, 2021-12-25}

Short rules: offline exam, 2 hours, closed book. 

\begin{enumerate}

    \item (10 points) Ded Moroz considers the function $f(x, y) = (1 + 2x + 3y)^{2022}$.
    
    Please, help him to find the second order Taylor approximation of this function at a point $x=0$, $y=0$.
    
    \item (10 points) Consider the function 
    \[
    f(x, y) = \int_0^x 3e^{u^2} \, du  + \int_0^y 2\cos(u^2) \, du.   
    \]
    
    Find the gradient $\grad f$ at the point $(0, 0)$.
    
    \item (10 points) Consider the system 
      \[
      \begin{cases}
        x + y + z = 2 \\
        2x^2 + 2y^2 = z^2 \\
      \end{cases}.  
      \]
      \begin{enumerate}
        \item Are the functions $x(z)$ and $y(z)$ defined in a neighborhood of the point $A(x=1, y=-1, z=2)$?
        \item Find $dx/dz$ at the point $A$ if possible. 
      \end{enumerate}
    
    \item (10 points)  
    The set $S$ is defined by $S = \{(x, y) \in \mathbb{R}^2 \mid 0 \leq y \leq 2- x^2\}$. 
    Two rectangles one on the top of the other are inscribed in $S$, 
    thus they have the common side and the upper vertices lie on this parabola.
    Let $A_1 + A_2$ be the sum of their areas, where $A_1 >0$, $A_2 >0$. 
    
    Consider the maximization problem $A_1 + A_2 \to \max$.
    \begin{enumerate}
      \item Solve the maximization problem or show that the maximum does not exist.
      \item Check whether the Weierstrass theorem is applicable.
    \end{enumerate}
    
    \item (10 points)  
    An implicit function is defined by equation $x^2 + y^2 + xy + 2x + 4y = 0$. Find all the point(s)
    $(x^*, y^*)$ on the curve represented by this equation, where
    \begin{enumerate}
      \item the tangent line to the curve is horizontal with the equation $y = c$;
      % \item this $c$ has the maximum possible value;
      \item conditions of the implicit function theorem are satisfied.
    \end{enumerate}
    
    \item (10 points)  
    Consider the function $g(x) = \sqrt{x-1+\sqrt{x-1+\sqrt{x-1+\sqrt{\cdots}}}}$ defined for $x>1$.
    
    \begin{enumerate}
        \item Express $g^{2}(x)$ through the linear combination of $g(x)$ and $x-1$.
        \item Find $g'(2)$.
    \end{enumerate}
    
    
    %\item (10 points) The turtle starts at the point $(0, 0)$. 
    %Every day she continues her way at constant speed in constant direction. 
    %Every midnight she drops her speed by 30\% and turns $90^{\circ}$ counterclockwise. 
    %On the first day she goes by 1 kilometer in the positive direction of $OX$ axis.
    
    %What is the ultimate destination of the turtle?
    
        
      \item (20 points) The Mean Value Theorem in Calculus claims that given a continuously differentiable function $f(x)$  
      on the closed segment $[x_0; x_0 + \Delta x]$, there exists $0 < \theta < 1$ such that $f(x_0 + \Delta x) - f(x_0) = 
      f'(x_0 + \theta \Delta x) \Delta x$. 
    
      Prove the version of this theorem for a function of the two variables $z = f(x,y) \in C^1$ in the following form: 
      there exists $0 < \theta < 1$ such that 
      \[
        f(x_0 + \Delta x, y_0 + \Delta y) - f(x_0, y_0) = \frac{\partial f}{\partial x}(x_0 + \theta \Delta x, y_0 + \theta \Delta y)\Delta x + \frac{\partial f}{\partial y}(x_0 + \theta \Delta x, y_0 + \theta \Delta y)\Delta y.  
      \]
      
      
      Hint. Join the 2 points $(x_0, y_0)$ and $(x_0 + \Delta x, y_0 + \Delta y)$ with the segment of a straight line passing through these points 
      and apply the chain rule to the composite function $f(x_0 + t \Delta x, y_0 + t \Delta y)$ where $t\in [0;1]$.
      
      
        \item An economist solves a problem of the minimization of expenses of an individual whose utility function is $u(x, y) = \sqrt{x} + \sqrt{y}$. 
      Expenses are calculated by the formula $E=3x + 4y$ and the consumer would prefer to fix the value of the utility at 7 utiles.
      \begin{enumerate}
      \item (5 points) Formulate the problem of the expenses minimization and form a Lagrangian of this problem.
      \item (5 points) Using first-order conditions find the minimizing bundle $(x^*, y^*)$.
      \item (5 points) Using bordered Hessian or otherwise check the sufficiency condition.
      \item (5 points) A consumer decided to improve her welfare by adding additional 0,1 to 7 utiles. 
      How this decision will affect the expenditure value $E^* = 3x^* + 4y^*$, 
      where the bundle $(x^*, y^*)$ corresponds to a greater utility value? 
      Use appropriate Envelope Theorem to estimate $E^*$.
    \end{enumerate}
    
    \end{enumerate}


\subsection{MFE, december exam, 2021-12-25, marking}

\begin{enumerate}
    \item Two first order and three second order derivatives: 5 points (each derivative: 1 point).
    
    Correct Taylor formula: 5 points. 

    Forgotten factor $2$ before $f''_{xy}$: penalty 2 points.

    Forgotten factor $1/2$: penalty 2 points.

    Taylor formula with powers higher than 2: penalty 4 points. 

    Standard solution:
    \[
    f(x, y) \approx f(0, 0) + f'_x(0, 0) x + f'_y(0, 0) y + 0.5 (f''_{xx}(0, 0) x^2 + 2f''_{xy}(0, 0) xy + f''_{yy}(0, 0) x^2).
    \]
    Non standard solution. Consider $h(t) = (1 + t)^{2022}$.
    \[
    h(t)  \approx h(0) + h'(0) \cdot t + 0.5 h''(0)\cdot t^2 = 1 + 2022t + 1011 \cdot 2021 t^2,
    \]
    where $t = 2x + 3y$.


    \item Knowledge of the gradient notion: 2 points. 
    
    Correct derivatives: 8 points (each derivative: 4 points).

    Common error, $3\exp(0^2)$ and $2\cos(0^2)$ are subtracted and everything else is ok, 
    total grade for the problem: 6 points.

    Derivative just cancels integral, so
    \[
        \grad f = (3 \exp(x^2), 2 \cos(y^2)) = (3, 2).
    \]
    
    \item IFT conditions: 5 points;
    
    Formula for derivative: 3 points;

    Final answer: 2 points.

    \[    
    x'(z) = - \frac{\det \begin{pmatrix}  
       \partial F_1/\partial z & \partial F_1 / \partial y \\
       \partial F_2/\partial z & \partial F_2 / \partial y \\
    \end{pmatrix}}{\det \begin{pmatrix}
        \partial F_1/\partial x & \partial F_1 / \partial y \\
        \partial F_2/\partial x & \partial F_2 / \partial y \\         
    \end{pmatrix}} = \frac{-4y + 2z}{4y-4x} = 0
    \]
    \item \begin{enumerate}
        \item  Correct introduction of the vertices coordinates with the statement of the maximization problem gives 2 points. 
    Setting the FOC and solving the system gives another 3 points. 
    Checking the Hessian sign definiteness is rewarded by 3 points. 
    \item The analysis of the set on which the objective function is maximized gives another 2 points for correct answer 
    or 1 point for the analysis of the set $S$. 
    This maximization problem should be solved not just on the set $S$, which is compact, but it should include conditions $A_1 >0$, $A_2 > 0$.
\end{enumerate}

\item

    
    $F(x,y):= x^2+y^2+xy+2x+4y = 0$
    
    The slope $y'(x)$ of the tangent line to the curve $y(x)$ should be equal to 0 \textit{(1 point).} 
    Using IFT we can find the derivative at the arbitrary point $(x_0,y_0)$:
    \[
    y'(x_0) = -\frac{\partial F(x_0,y_0)/\partial x }{\partial F(x_0,y_0)/\partial y}
    \]
    \textit{(1 point)}
    
    However, do not forget to mention the IFT conditions (here or in b) that these points should satisfy:
    \begin{itemize}
        \item $F(x,y)\in C^1$ (continuosly differentiable) as a polynomial function;
        \item $\partial F(x_0,y_0)/\partial y = 2y+x+4 \neq 0$.
    \end{itemize}\textit{(3 points)}
    
    \[
      y'(x) = -\frac{2x+y+2}{2y+x+4} = 0.
    \] 
    
    Thus, to find the requested point we need to solve the system of equations 
    \[
    \begin{cases}
        y = -2x-2,\\
        x^2+y^2+xy+2x+4y = 0
    \end{cases}
    \]

The solution is $(2,-6)$ and $(-2/3,-2/3).$ \textit{(3 points)}

To complete b) we need to check if these points satisfy the conditions of the IFT. 
Verify that $2y + x +4 \neq 0$ at both of them or prove it using any other appropriate method. \textit{(2 points)} 
    
\textit{Please note that solutions without substituting $y = -2x-2$ to the initial identity $F(x,y) = 0$ can gain 5 points tops.}

\item

\begin{enumerate}
    \item Raise the function to the second power:
    
    \[
        g^2(x) = g(x) + (x-1)
    \]
    
    \textit{(3 points)}

    
    \item Find the derivative of both sides in a):
    \[
        2g(x)\cdot g'(x) = g'(x) + 1
    \]
    Thus, 
    \[
      g'(x) = \frac{1}{2g(x)-1}
    \]
    Therefore,
    \[
      g'(2) = \frac{1}{2g(2)-1}
    \]
    \textit{(4 points)}
    
    The only thing left is to find $g(2)$. Use a):
    \[
    g^2(2) - g(2) - 1 = 0
    \]
    Solve the quadratic equation and use the fact that $g(2) >0$.
    
    \[
      g(2) = \frac{1+\sqrt{5}}{2}
    \]
    \textit{(3 points)}
\end{enumerate}
    

    \item 
    \item 
\end{enumerate}
