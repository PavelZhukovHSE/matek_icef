% !TEX root = ../matmor_exams.tex

\subsection{MFE, October mock, 2020-10-24}


\begin{enumerate}

\item (10 points) Consider the function $f(x, y) = x^3 +3a y^3 + 2axy$ where $a$ is a parameter. 
  Using the total differential find the approximate value of $f(1.98, 0.99)$.

  \item (10 points) Consider the system
  \[
  \begin{cases}
  3x^3 + u(y) + u(z) = 5 \\
  x + x^3 + 2u(y^3x) = 4 \\
  \end{cases},
  \]
  where $u(x)$ is a differentiable function with $u(1)=1$. 


  \begin{enumerate}
    \item Clearly state conditions sufficient to guarantee that the system defines the functions $z(y)$ and $x(y)$ at a point $(1, 1, 1)$. 
    \item Find $z'(y)$ provided the conditions are met.
 \end{enumerate}

 \item (10 points) Consider the function $f (x, y) = xy^3$  and the point $A = (-1, -2)$.

 Find the direction of the maximal rate of change of the function and this maximal rate.


\item (10 points) For $x>0$, $y>0$ find the limit 
\[
u(x,y) = \lim_{t\to 0} \left(x^{\frac{t-1}{t}} + y^{\frac{t-1}{t}}  \right)^{\frac{t}{t-1}}
\]

\item (10 points) Provide an explicit example of a non-convergent sequence $(x_n)$ in $\RR^2$ 
such that the sequences $y_n = \| x_n \|$ and $z_n = \| x_n  + 2x_{n+1}\|$ are convergent.

\item (10 points) Let's consider the sets $A_n = \{ x\in \RR \mid x^2 n = 1\}$.

Describe the set $A = \cup_{n=1}^{\infty} A_n$: is it closed, open, bounded, compact?



\item 
A curve represented by the equation $(x^2 + y^2)^2 = x^2 - y^2$ is called lemniscate. 

\begin{enumerate}
  \item (5 points) While solving for $y=y(x)$ implicit function defined by this equation 
  is it possible to use IFT in the neighborhood of the point $(0,0)$?
  \item (5 points) Show that in the first quadrant $\{(x,y) \mid x>0, y>0 \}$ 
  such implicit function $y=y(x)$ exists. Justify your reasoning. 
  \item (10 points) Find $y'(0)$ if it exists. 
\end{enumerate}

Hint: for c) it is convenient  to change Cartesian coordinates to polar coordinates 
following the formulas $x=r\cos \phi$ and $y=r\sin \phi$.


\item Let $u$ be a composite function $u=g(x^2 + y^2)$, where $g(t) \in C^2$ for $t>0$. 

\begin{enumerate}
  \item (5 points) Is the formula 
  \[
du = g'(x^2 + y^2)(2xdx + 2ydy)
  \]
  for the total differential valid? Provide a clear argument. 
\item  (5 points) For higher order differentials we would like to continue in the same fashion:
\[
d^2u = g''(x^2 + y^2)(2xdx + 2ydy)^2
\]  
Does this method work? Justify your answer. 
\item (10 points) Let $g(t)=\sqrt{t}$. 
Prove that for the function $u(x,y)=\sqrt{x^2 + y^2}$ the second-order differential is non-negative.
\end{enumerate}

  


  
\end{enumerate}
  

\subsection{MFE, October mock, 2020-10-24, marking}


\begin{enumerate}

    \item
    The total differential at point $(2,\,1)$ is given by $df=f'_x(2,\,1)dx+f'_y(2,\,1)dy$. The partial derivatives are $f'_x(x,\,y)=3x^2+2ay$ and $f'_y(x,\,y)=9ay^2+2ax$, so $f'_x(2,\,1)=12+2a$ and $f'_y(2,\,1)=13a$. Therefore, given that $\Delta x = 1.98-2 = -0.02$ and $\Delta y=0.99-1=-0.01$, we have
    \[\begin{split}
    f(1.98,\,0.99)\approx f(2,\,1)+f'_x(2,\,1)\Delta x+f'_y(2,\,1)\Delta y=(8+7a)+(12+2a)(-0.02)+13a(-0.01)=\\=7.76+6.83a
    \end{split}
    \]
    
    Marking:
    \begin{itemize}
        \item  $f'_x(2,1)$ and $f'_y(2,1)$ calculated -- 1 point each,
        \item correct (negative) $\Delta x$ and $\Delta y$ -- 1 point each,
        \item correct formula for $f(x_0+\Delta x,\,y_0+\Delta y)$ -- 4 points,
        \item correct final answer -- 2 points.
    \end{itemize}
    
    Frequent mistakes:
    \begin{itemize}
        \item used the starting point $(1,\,1)$ -- penalty of 1 point.
    \end{itemize}
    
    \item Let $F(x,\,y,\,z)=3x^2+u(y)+u(z)$ and $G(x,\,y,\,z)=x+x^3+2u(y^3x)$.
    \begin{enumerate}
        \item The Implicit Function Theorem has 3 conditions to be verified:
        \begin{enumerate}
            \item The equations are satisfied at the point considered. This is indeed true --
            \[
            \begin{cases}
                F(1,\,1,\,1)=3\cdot 1^2+u(1)+u(1)=3+1+1=5,\\
                G(1,\,1,\,1)=1+1^3+2u(1^3\cdot1)=1+1+2=4.
            \end{cases}
            \]
            \item The function is continuously differentiable in a neighbourhood of the point considered. $F$ and $G$ are sums of polynomials (which are continuously differentiable) and a differentiable function $u$. For the conditions of the theorem to be satisfied, the function $u(x)$ has to be continuously differentiable in a neighbourhood of $x=1$.
            \item The determinant of the Jacobian $J=\frac{\partial(F,\,G)}{\partial(x,\,z)}$ is not 0 at the point considered. Checking:
            \[
            |J|=\begin{vmatrix}\frac{\partial F}{\partial x}& \frac{\partial F}{\partial z}\\
            \frac{\partial G}{\partial x}& \frac{\partial G}{\partial z}\end{vmatrix}=
            \begin{vmatrix}9x^2&u'(z)\\1+3x^2+2y^3u'(y^3x)&0\end{vmatrix}=-u'(z)(1+3x^2+2y^3u'(y^3x))
            \]
            At $x=y=z=1$ we have $|J|=-u'(1)(4+2u'(1))$. For $|J|\neq 0$ we should have $u'(1)\neq 0$ and $u'(1)\neq -2$.
            \end{enumerate}
        Marking:
        \begin{itemize}
        \item verified that the point lies on the curve -- 1 point
        \item any justification that the functions are differentiable (not necessarily continuously) -- 1 point
        \item correct expression for the Jacobian -- 1 point,
        \item computing all the partial derivatives in the Jacobian -- 1 point
        \item computing the determinant of the Jacobian -- 1 point
        \item answer in terms of $u'(1)$ -- 1 point.
        \end{itemize}
    \item Assuming the conditions in part (a) hold, the IFT gives the expression for $z'(y)$:
    \[\begin{split}
    z'(y)=-\frac{\begin{vmatrix}\frac{\partial F}{\partial x}& \frac{\partial F}{\partial y}\\
            \frac{\partial G}{\partial x}& \frac{\partial G}{\partial y}\end{vmatrix}}
            {\begin{vmatrix}\frac{\partial F}{\partial x}& \frac{\partial F}{\partial z}\\
            \frac{\partial G}{\partial x}& \frac{\partial G}{\partial z}\end{vmatrix}}=
            -\frac{\begin{vmatrix}9x^2&u'(y)\\1+3x^2+2y^3u'(y^3x)&6y^2xu'(y^3x)\end{vmatrix}}{-u'(z)(1+3x^2+2y^3u'(y^3x))}=\\=-\frac{54x^3y^2u'(y^3x)-u'(y)(1+3x+2y^3u'(y^3x))}{-u'(z)(1+3x^2+2y^3u'(y^3x))}
    \end{split}
    \]
    Note that in order to get the actual dependence $z'(y)$, $x$ and $z$ in the latter expression should be understood as $x(y)$ and $z(y)$ correspondingly.
    Marking:
    \begin{itemize}
        \item correct formula for $z'(y)$ -- 1 point,
        \item computing all partial derivatives in it -- 1 point,
        \item final answer -- 2 points.
    \end{itemize}
\end{enumerate}

    Frequent mistakes:
    \begin{itemize}
    \item the IFT is written down, but its requirements are not verified -- no points for the requirements which have not been verified,
    \item $u$ considered as a separate variable instead of a function -- no more points after this mistake,
    \item $u$ considered as a function of several variables (via partial derivatives $u'_x,\,u'_z$) -- no more points after this mistake,
    \item incorrect formulas for the Jacobian and/or for $z'(y)$ -- mistaken variables and/or forgotten minus sign -- no more points after this mistake,
    \item found $z'(1)$ instead of $z'(y)$ -- penalty of $1$ point.
    \end{itemize}

        
    \item direction of maximal growth rate — 6 points, maximal growth rate — 4 points. 

    Frequent errors:
    
    No vector, only derivatives are calculated — 2 points.
    
    Gradient is computed but is not considered as the best direction — 4 points.

    \item $y/x$ lives matter 4 points. 
    
    \item Probably, the simplest example is something like $x_n = (a_n, b_n),$ where $a_{n} = 1,$ $b_{n} = -1$ for odd $n$ and $a_{n} = -1,$ $b_n = 1$ for even $n.$ Indeed, $\forall n$ we have $||x_n|| = \sqrt{2}$   and $||x_n + 2x_{n+1}|| = \sqrt{2}$ which are obviously convergent. At the same time the sequence $x_n$ itself is divergent. 
    However, there are infinitely many of such examples. All of them, if correctly explained, have been awarded the maximum. 3 points were awarded if the problem was solved for $x_n \in \mathbb{R}.$   
    \item  Note, that the set is just a combination of the two sequences: 

    $$A = \{\pm 1, \pm \frac{1}{\sqrt{2}}, \pm \frac{1}{\sqrt{3}}\dots\}
    $$
    All the points of $A$ are boundary points (which makes the set obviously \textbf{not open}). However, there is a boundary (an accumulation) point $0$ which is not included in $A.$ Thus, the set is \textbf{not closed} (closed set should contain all its boundary points). As the set is not closed then it is \textbf{not compact} (compact set should be closed and bounded). The set is obviously \textbf{bounded}: for every $x\in A$ we have $|x|\leq 1.$ The end.
    
    Marking:
    \begin{itemize}
        \item  Explain why the set is not closed - 3 points
        \item Explain why the set is not open - 2 points
        \item  Explain why the set is bounded - 3 points
        
        \item  Explain why the set is not compact - 2 points (1 point was awarded for the definition of compact)
        
        
    \end{itemize}
  
    
    
    \item Problem 7
    
    a) IFT is not applicable - then 6 pts
    
    b) Correct explanation - 6 pts
        
        Partially correct explanation or graphical illustration or IFT reapplied - 3 pts
    
    c) Polar coordinates substitutaion - 5 pts
        
        Full proof - 5 pts

    \item Problem 8
    
    a) any sustainable proof or explanation that formula is correct - 6 pts
    
    b) any sustainable proof or explanation that formula NOT correct - 4 pts
    
    c) 2nd order derivative computed - 5 pts
    
        full proof - 5 pts


\end{enumerate}


\subsection{December midterm, 2020-12-25}

Comments: 2 hours, online with proctoring. 


\begin{enumerate}

\item (10 points) Consider the function $f(x, y) = x^3 + 3y^2 x^2$ and the point $A(1,1)$.
\begin{enumerate}
  \item Calculate the gradient of the function $f$ at the point $A$.
  \item Find the second order Taylor approximation in the neighborhood of $A$.
\end{enumerate}

\item (10 points) Consider the equation $3x^3 + 5y^5 + z^3 + z=10$. 
\begin{enumerate}
  \item Check whether the equation defines the function $z(x, y)$ at a point $A(1,1,1)$.
  \item Find $dz$ at the point $A$.
\end{enumerate}


\item (10 points) Find all local extrema of the function $f(x, y) = x^2 y - 3xy^2 + 5xy +2$ such that $x\neq 0$ and $y\neq 0$. 
Classify them.

\item (10 points) Find all local constrained extrema 
of the function $f(x, y, z) = x + 2y + 3z$ subject to $\ln x + \ln y + \ln z = 0$.
Do not forget to classify extrema. 

\item (10 points) Consider the function $f(x, y) = h(x) g(y) + ax^3$, where $h$ and $g$ are twice differentiable
and $a$ is a parameter. Let's denote the maximum point by $x^*(a)$ and $y^*(a)$ and assume that second order conditions for maximization are met.

Find the sign of $dx^*/da$. 

\item (10 points) The level curves of the function $f(x,y)$ are given by the equation $y - x^2 = c$.

Draw two level curves of the function $g(x,y)=f(x-2, |y| + 1)$.

\item Consider a problem 
\[
  \begin{cases}
xyz \to \max \\
\text{s.t. } x+ y+ z = c \\
x, y, z >0 
  \end{cases}
\]
where $c$ is a parameter and $c>0$.
\begin{enumerate}
  \item (15 points) Solve this problem using first-order conditions. Use bordered
Hessian for sufficiency.
\item (5 points) Use the result of part a) to show that arithmetic mean 
$(x + y + z)/3$ is 
no less than the geometric mean  $(xyz)^{1/3}$.
\end{enumerate}

\item 
\begin{enumerate}
  \item (10 points) Consider a utility maximization problem
\[
  \begin{cases}
    u(x,y) \to \max \\
    \text{s.t. } p_x x + p_y y  = I \\
      x, y > 0 \\
  \end{cases},
\]
where $u\in C^1$ and parameters $p_x$, $p_y$ and $I$ are positive. 
Let $(x^*, y^*)$ be the solution of this problem. Form the value function
$V(p_x, p_y, I) = u(x^*, y^*)$. 

Using appropriate envelope theorem show that 
\[
  x^* = - \frac{\partial V}{\partial p_x} / \frac{\partial V}{\partial I}, \quad 
  y^* = - \frac{\partial V}{\partial p_y} / \frac{\partial V}{\partial I}.
\]

% (so called Roy’s identity).

\item (10 points) Let $F(x,y)$ be a function such that $F\in C^2$ and $F'_y  \neq 0$. 
The equation $F(x, y)=0$ defines the implicit function $y(x)$.

Find the expression for $d^2 y / dx^2$.

The expression should include only derivatives of $F(x,y)$.
\end{enumerate}


\end{enumerate}

    



\subsection{December midterm marking scheme}

\begin{enumerate}
    \item
    \begin{enumerate}
        \item The gradient of $f$ is a vector $\begin{pmatrix}f'_x\\f'_y\end{pmatrix}$. Calculating derivatives, 
        we get $f'_x(x,y)=3x^2+6xy^2$ and $f'_y=6x^2y$. 
        So at the point $A(1,1)$ the gradent $\nabla f=\begin{pmatrix}9\\6\end{pmatrix}$.
        
        Marking: 1 point for each derivative (as a formula). 1 point for each correct coordinate of the answer.
        
        \item The second-order Taylor approximation in a neighbourhood of $(x_0,y_0)$ 
        is given by 
        \begin{multline*}
            T_2(x,y)=f(x_0,y_0)+f'_x(x_0,y_0)(x-x_0)+f'_y(x_0,y_0)(y-y_0)+\\
            +\frac12[f''_{xx}(x-x_0)^2+2f''_{xy}(x-x_0)(y-y_0)+f''_{yy}(y-y_0)^2].
        \end{multline*}
        Calculating derivatives, we get $f''_{xx}=6x^2+6y^2$, $f''_{xy}=12xy$, $f''_{yy}=6x^2$, 
        so at the point $A(1,1)$ $f''_{xx}=12,\,f''_{xy}=12,\,f''_{yy}=6$. Therefore,
        \[
        T_2(x,y)=4+9(x-1)+6(y-1)+\frac12[12(x-1)^2+2\cdot12(x-1)(y-1)+6(y-1)^2].
        \]
        
        Marking: 1 point for each of 3 derivatives (as a number, not as a formula), 3 points for the answer. 
        Penalties: -1 if the answer is in the paper, but is not final 
        (e.g. something else is presented as an answer and it's not just algebraic transforms like opening the brackets),
        -1 for forgetting $f(x_0,y_0)$, -1 for having the linear part wrong. -1 for having the quadratic part wrong.
    \end{enumerate}
    \item
    \begin{enumerate}
        \item The Inverse function theorem requires that:
        \begin{itemize}
            \item $F(1,1,1)=0$ — this is correct as $3\cdot1^3+t\cdot1^5+1^3+1=10$;
            \item $F$ is continuously differentiable — this is correct as $F$ is a polynomial;
            \item $F'_z(1,1,1)\neq 0$ — this is true as $F'_z(x,y,z)=3z^2+1>0$.
        \end{itemize}
        Therefore, by the IFT, the equation does define a function $z(x,y)$ in a neighbourhood of $(1,1,1)$.
        
        Marking: 1 point for each condition correctly verified, 1 point for stating the name of the theorem, 
        1 point for the correct conclusion (only if all 3 conditions are correct). 
        Note that part b) is not considered when grading part a).
        
        \item From the IFT we have 
        \[
        z'_x(1,1)=-\frac{F'_x(1,1,1)}{F'_z(1,1,1)}=-\frac94;\, z'_y(1,1)=-\frac{F'_y(1,1,1)}{F'_z(1,1,1)}=-\frac{25}{4}.
        \]
        Therefore, at point $A$
        \[
        dz=z'_x\,dx+z'_y\,dy=-\frac94\,dx-\frac{25}{4}\,dy.
        \]
        
        Marking: 1 point for $z'_x(1,1)$ and $z'_y(1,1)$ (as numbers). 3 points for the answer. 
        Penalties: -1 if the point $A$ is not substituted, -1 if the answer is in the paper, but is not final, 
        -2 for writing $(x-1)$ and $(y-1)$ instead of $dx$ and $dy$.
        
        
        
        
    \end{enumerate}

        \item FOC 3 — pts, coordinates of a point — 2 pts, Hessian  — 2 pts, Point classified correctly — 3 pts.

        \item NDCQ — 1 pt, FOC — 2 pts, Point — 2 pts, Hessian — 3 pts, Point classified correctly — 2 pts.

        \item FOC — 2 points. Formula for the implicit derivative with two determinants — 2 points. 
        The sign of the denominator — 3 points, the sign of the nominator — 3 points.
        
        Short solution for $f(x,y) = h(x)g(y) + ax^3$.
        
        FOC:
        \[
        \begin{cases}
        h'(x)g(y) + a 3x^2 = 0 \\
        h(x)g'(y) = 0
        \end{cases}
        \]
        
        Formula for the derivative:
        \[
        \frac{dx^*}{da} = - \frac{\det A}{\det B}
        \]
        
        Here $\det B$ will be exactly the determinant of the Hesse matrix, so it is positive by SOC.
        
        And $\det A = 3x^2 h(x) g''(y)$. Once again we use SOC and we get $\det A < 0$.
        
        \item Two clearly stated equations of level curves — 4 points. Plots of these two equations — 6 points.
        
        Typical errors: wrong reflection for $|y|$ — 3 points penalty; only one level curve is drawn — 2 points penalty.
        
        \item 
        \begin{enumerate}
            \item NDCQ - 2 points, FOC - 3 points, FOC solution - 5 points, SOC solution - 5 points
            \item clear explanation of inequality case - 5 points
        \end{enumerate}
        \item
        \begin{enumerate}
            \item Envelope Theorem in general case — 4 points, using Theorem about the meaning of Lagrange multiplier or application of general case of ET to find it — 4 points, derivation the formulas - 2 points
            \item  first derivative — 4 points, quotient rule for calculation of the second derivative - 2 points, correct application of chain rule - 2points, rewriting $y’$ in terms of partial derivatives of F — 2 points.
        \end{enumerate}

\end{enumerate}



\subsection{April, 2021-04-03}

Time: 120 minutes, offline exam. 

\begin{enumerate}
  \item (10 points) Solve the difference equation $x_n - 8 x_{n-1} + 16 x_{n-2} = 2^n$.

  \item (10 points) Solve the differential equation $y'' - 8 y' + 17 y = \cos t$.
\item (10 points) 
Find the second order Taylor approximation at a point $(0, 0)$ of the function 
\[
  f(x, y) = \cos(x + 3y) - \sin (3x + y).
\]

\item (10 points) Solve the optimization problem 
\[
2 x_1 + 3 x_2 + 4 x_3 \to \max  
\]
subject to
\[
\begin{cases}
x_1 \geq 0,  x_2 \geq 0, x_3 \geq 0  \\
2 x_1 + 2 x_2 + 3x_3 \leq 12 \\
3 x_1 + 3 x_2 + 2x_3 \leq 12 \\ 
\end{cases}.
\]

\item (10 points) Solve the optimization problem 
\[
x^2 + y^2 + z^2 - 16x - 16y + 12 z \to \min
\]
subject to
\[
x + y \leq 15. 
\]

\item (10 points) Sketch the set $A = \{z \mid z^2 + 2z \in \mathbb R\}$ on the complex plane $\mathbb C$.

\item Let $F(K, L)$ be twice continuously differentiable function with positive derivatives $F'_K >0$, $F'_L > 0$ for all $K > 0$ and $L > 0$. 
The function $F$ is homogeneous of degree 1.
\begin{enumerate}
    \item (10 points) Prove that the determinant of its Hessian matrix is 0 for all $K > 0$ and $L > 0$.
    \item (10 points) Let $Y = F(K, L)$. Denote the derivatives with respect to time by $\dot Y$, $\dot K$, $\dot L$. 
    Prove that there exists a function $0 < \alpha(t) < 1$, such that 
    $\dot Y/Y = \alpha(t) \dot L/L + (1-\alpha(t)) \dot K/K$.
\end{enumerate}


\item (20 points) Consider the second order differential equation with constant coefficients $y'' + a y' + by = 0$. 
Let initial values $y(0)^2 + (y'(0))^2 \neq 0$. 

Find conditions (necessary and sufficient) on the coefficients $a$ and $b$ 
that guarantee that every solution of this equation in absolute value $\abs{y(x)}$ 
will monotonically increase starting with some $x_0$.


\end{enumerate}

\subsection{April Exam Marking Scheme}
\begin{enumerate}
  \item %1
  \item %2
  \item %3
      Derivatives: $f'_x(0,\,0)=-3,\,f'_y(0,\,0)=-1,\,f''_{xx}(0,\,0)=-1,\,f''_{xy}(0,\,0)=-3,\,f''_{yy}(0,\,0)=-9$ -- 1 point each. No points if the values at $(0,\,0)$ are not evaluated.
      
      $T_2(x,\,y)=1-3x-y+\frac12(-x^2-2\cdot3xy-9y^2)$ -- 5 points. Only 3 points for forgotten $2$ or $\frac12$.
  \item %4
      Correct dual problem -- 2 points. Only 1 for forgotten non-negativity constraints for dual variables. No points for the entire problem if the dual problem is otherwise incorrect.
      
      Solved the dual problem -- 3 points. Only 1 if the choice of the optimal point is not proven (either via slope considerations or by considering all corner points). Only 1 point for `we can find the optimal point from the graph' or other incomplete reasonings.
      
      Optimal value for the primal problem stated -- 2 points.
      
      Optimal $x_{1,2,3}$ found -- 3 points.
  \item %5
  \item %6
  \item %7
  \item %8
\end{enumerate}