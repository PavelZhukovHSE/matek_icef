\documentclass{article}
\usepackage{amsmath, amsfonts, amsthm}
\usepackage[cp1251]{inputenc}
\usepackage[russian]{babel}
\usepackage{lscape}
\usepackage{amssymb}
\voffset -1in
\hoffset -12mm
\pagestyle{empty}
\topmargin=1cm

\oddsidemargin=5mm
\textwidth = 17cm
\textheight=25cm
\usepackage[cp1251]{inputenc}
\usepackage[russian]{babel}
\begin{document}

\fontsize{14}{21}
\selectfont
\centerline{\textbf{Assignment 13 (Due on the week December 15 -- 20)}}
\fontsize{12}{18}
\selectfont
\begin{enumerate}
\item Let $f(x,y)=-x^2-y^2$, and we seek to maximize that function subject to constraint $(x-1)^3=y^2$. Solve that problem with and without the additional Lagrange multiplier $\lambda_0$.

\item Find the critical points in the problem of constrained optimization and classify them using the second-order conditions: $f(x,y,z)=xyz\rightarrow extr$ , subject to $x^2+y^2+z^2=1$, $x+y+z=0$.

\item The weekly production of a factory depends on the amounts of capital and labor it employs by the formula $q(k,l)=\sqrt{kl}$. The cost of capital is \$4 per unit and the cost of labor is \$1. Find the minimum weekly cost of producing $q=200$. How the cost of production changes if the factory has to produce $q=202$?

\item A firm's inventory $I(t)$ is depleted at a constant rate per unit time, i.e. $I(t)=x-\delta t$, where $x$ is an amount of good reordered by the firm whenever the level of inventory is zero. The order is fulfilled immediately. The annual requirement for the commodity is $A$ and the firm orders the commodity $n$ times a year where $A=nx$. The firm incurs two types of inventory costs: a holding cost and an ordering cost. Since the average stock of inventory is $x/2$ the holding cost equals $C_hx/2$, the cost of placing one order is $C_0$ and with $n$ orders a year this cost equals $C_0n$.
\begin{enumerate} 
\item Minimize the cost of inventory $C=C_hx/2+C_0n$ by choice of $x$ and $n$ subject to the constraint $A=nx$ by the Lagrange multiplier method.
\item Using the envelope theorem interpret the Lagrange multiplier.
\end{enumerate}
\item Use the Lagrange multipliers to find the dimensions of a rectangular box with the least possible surface area among those with a volume of 27 m$^3$. Check the second-order conditions. Evaluate the change in the minimal surface area if the volume drops by 0.5 m$^3$. Compare your estimate with the direct computation.
\end{enumerate}
\end{document}

