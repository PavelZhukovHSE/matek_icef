\documentclass{article}
\usepackage{amsmath, amsfonts, amsthm}
\usepackage[cp1251]{inputenc}
\usepackage[russian]{babel}
\usepackage{lscape}
\usepackage{amssymb}
\voffset -1in
\hoffset -12mm
\pagestyle{empty}
\topmargin=1cm

\oddsidemargin=5mm
\textwidth = 17cm
\textheight=25cm
\usepackage[cp1251]{inputenc}
\usepackage[russian]{babel}
\begin{document}

\fontsize{14}{21}
\selectfont
\centerline{\textbf{Assignment 20 (Due on the week March 10 -- 14)}}
\fontsize{12}{18}
\selectfont
\begin{enumerate}
\item For each of the following difference equations find the complementary function, the particular integral and the definite solution:

\begin{enumerate}
\item $y_{t+1}+3y_t=4\; (y_0=4)$
\item $2y_{t+1}-y_t=6\; (y_0=7)$
\item $y_{t+1}=0.2y_t+4\; (y_0=4)$
\end{enumerate}

\item Find the solutions of the following equations and determine whether the time paths are oscillatory and convergent:
\begin{enumerate}
\item $y_{t+1}-\frac13y_t=6\; (y_0=1)$
\item $y_{t+1}+2y_t=9\; (y_0=4)$
\item $y_{t+1}+\frac14y_t=5\; (y_0=2)$
\item $y_{t+1}-y_t=3\; (y_0=5)$
\end{enumerate}

\item Solve graphically the following linear program:
$$
\begin{cases}
\mathrm{max}(4x_1+2x_2)\\
2x_1+3x_2\leqslant 18,\\
-x_1+2x_2\leqslant 9,\\
2x_1-4x_2\leqslant 10,\\
x_1\geqslant 0,\, x_2\geqslant 0.
\end{cases}
$$

\item Find the minimal value of the objective function in the linear program stated below, where $\lambda$ is a real value parameter using the dual program

$$
\begin{cases}
\mathrm{min} (x_1+10 x_2+2x_3+4x_4),\\
\lambda x_1+2x_2-x_3+x_4\geqslant -1,\\
x_1+3x_2+x_3+x_4\geqslant 5,\\
x_i\geqslant 0,\; i=1,2,3,4.
\end{cases}
$$

\item Consider the utility maximization problem in the economy of $n$-goods:\\ $U(x_1,x_2,\dots, x_n)=x_1^{a_1}x_2^{a_2}\dots x_n^{a_n}\rightarrow \mathrm{max}$ subject to the budget constraint $\sum\limits_{i=1}^nx_i\leqslant I$, $x_i\geqslant 0$, $a_i>0$.
Find the optimal bundle an check whether you found the maximum. Find the rate of change of the indirect utility function when the income slightly changes.
\end{enumerate}
\end{document} 
